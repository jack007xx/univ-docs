
\section{実験2-1}
\subsection{実験の目的,概要}
本実験では、ディスプレイに文字'B'を繰り返し表示するMIPSマシンコードを、実験1-2で作成したプロセッサ上で動作させる。
その際,事前にプロセッサの動作について予想しておき,結果と比較する。

これによって、実験1-1と同様にプロセッサ上で足りない命令を確認することを目的とする。

\subsection{実験方法}
\subsubsection{メモリイメージファイルの作成}
中身が以下のようなバイナリファイルprint\_B\_while.binを配置した。
\lstinputlisting[caption=print\_B\_while.bin,label=printBwhile.bin]{src/obj2-1/print_B_while}

以下のコマンドでROM配置用のイメージファイルと,機能レベルシュミレーション用のverilog HDL記述ファイルを生成した。
\begin{lstlisting}[caption={イメージファイルの作成},label={イメージファイルの作成2-1}]
  $ bin2v print_B_while.bin
\end{lstlisting}

\subsubsection{命令列の確認}
生成されたrom8x1024\_sim.vを参照し,以下の点について結果を予測した。
\begin{itemize}
  \item プロセッサが PC=0x004c の命令を実行することにより,PC に格納される値と,それが表す命令メモリの番地。
\end{itemize}

\subsubsection{論理合成}
以下を実行し,mips\_de10-lite.tar.gzを解凍し,その中にメモリのイメージファイルを配置してコンパイルした。
\begin{lstlisting}[caption={論理合成操作},label={論理合成操作2-1}]
  $ mv rom8x1024.mif mips_de10-lite/
  $ cd mips_de10-lite/
  $ quartus_sh --flow compile MIPS_Default
\end{lstlisting}

\subsubsection{FPGAでの回路実現}
ACケーブル,モニタへのVGAケーブル,PCへのUSB接続,の順に行なったのち,モニタ電源をオンにした。
以下のコマンドで,PCでの接続状況を確認し,FPGAにダウンロードした。
\begin{lstlisting}[caption={FPGAでの回路実現},label={FPGAでの回路実現2-1}]
  $ dmesg
  $ quartus_pgm MIPS_Default.cdf
\end{lstlisting}

スイッチ0,1をonにして手動モードに切り替えた。
KEY0を押し,プロセッサをリセットした。
KEY1でカウンタを進めて,予想した点について動作を確認した。

\subsection{実験結果}
\subsubsection{メモリイメージファイルの作成}
\ref{イメージファイルの作成2-1}の操作によって,rom8x1024\_sim.vと,rom8x1024.mifが生成された。
内容は以下のように,ROMへのアクセスと応答をエミュレーションしたシュミレーションファイルと,実際に命令列を配置するためのイメージファイルだった。

\lstinputlisting[caption=rom8x1024\_sim.v,label=rom8x1024sim.v2-1]{src/obj2-1/rom8x1024_sim.v}
\lstinputlisting[caption=rom8x1024.mif,label=rom8x1024.mif2-1]{src/obj2-1/rom8x1024.mif}


\subsubsection{命令列の確認}
rom8x1024\_sim.vを参照し,以下のように予想した。
\begin{itemize}
  \item プロセッサが PC=0x004c の命令を実行することにより,PC に格納される値と,それが表す命令メモリの番地。
  \begin{itemize}
    \item PC=0x002cで,番地も0x002c
  \end{itemize}
\end{itemize}

\subsubsection{論理合成,FPGAでの回路実現}
プログラムカウンタの値は,通常通り0x0050になった。

\subsection{考察}
