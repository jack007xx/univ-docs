
\section{実験1-1}
\subsection{実験の目的,概要}

\subsection{実験方法}
\subsubsection{メモリイメージファイルの作成}
中身が以下のようなバイナリファイルprint\_B.binを配置した。
\lstinputlisting[caption=print\_B.bin,label=printB.bin]{src/01/print_B}

以下のコマンドでROM配置用のイメージファイルと,機能レベルシュミレーション用のverilog HDL記述ファイルを生成した。
\begin{lstlisting}[caption={イメージファイルの作成},label={イメージファイルの作成1-1}]
  $ bin2v print_B.bin
\end{lstlisting}

\subsubsection{命令列の確認}
生成されたrom8x1024\_sim.vを参照し,以下の点について結果を予測した。
\begin{itemize}
  \item PC=0x0040 002cの命令を実行したときの,REG[2]の値
  \item PC=0x0040 0030の命令を実行したときの,RAMの768番地の値
  \item PC=0x0040 0034の命令を実行したときの,REG[3]の値
  \item PC=0x0040 003cの命令を実行したとき,RAMの何番地の値がどう変化するか
  \item PC=0x0040 0048の命令を実行したとき,RAMの何番地の値がどう変化するか
\end{itemize}

\subsubsection{論理合成}
以下を実行し,mips\_de10-lite.tar.gzを解凍し,その中にメモリのイメージファイルを配置してコンパイルした。
\begin{lstlisting}[caption={論理合成操作},label={論理合成操作1-1}]
  $ tar -xvfz ./mips_de10-lite.tar.gz
  $ mv ./rom8x1024.mif ./mips_de10-lite/
  $ cd mips_de10-lite/
  $ quartus_sh --flow compile MIPS_Default
\end{lstlisting}

\subsubsection{FPGAでの回路実現}
ACケーブル,モニタへのVGAケーブル,PCへのUSB接続,の順に行なったのち,モニタ電源をオンにした。
以下のコマンドで,PCでの接続状況を確認し,FPGAにダウンロードした。
\begin{lstlisting}[caption={FPGAでの回路実現},label={FPGAでの回路実現1-1}]
  $ dmesg
  $ quartus_pgm MIPS_Default.cdf
\end{lstlisting}

スイッチ0,1をonにして手動モードに切り替えた。
KEY0を押し,プロセッサをリセットした。
KEY1でカウンタを進めて,予想した点について動作を確認した。

\subsection{実験結果}
\subsubsection{メモリイメージファイルの作成}
\ref{printB.bin}の操作によって,rom8x1024\_sim.vと,rom8x1024.mifが生成された。
内容は以下のように,ROMへのアクセスと応答をエミュレーションしたシュミレーションファイルと,実際に命令列を配置するためのイメージファイルがだった。

\subsubsection{命令列の確認}
rom8x1024\_sim.vを参照し,以下のように予想した。

\begin{description}
  \item [PC=0x0040 002cの命令を実行したときの,REG2の値]\mbox{}\\
    ゼロレジスタに768が加算されるので,768(=0x00000300)
  \item [PC=0x0040 0030の命令を実行したときの,RAMの768番地の値]\mbox{}\\
  ゼロレジスタの値が代入されるので,0
  \item [PC=0x0040 0034の命令を実行したときの,REG3の値]\mbox{}\\
    772
  \item [PC=0x0040 003cの命令を実行したとき,RAMの何番地の値がどう変化するか]\mbox{}\\
    \$3→772で,\$2→2なので,RAM[772]=2になる
  \item [PC=0x0040 0048の命令を実行したとき,RAMの何番地の値がどう変化するか]\mbox{}\\
    00400030でRAM[\$2→768]=0なので,RAM[768]が0→1になる
\end{description}

\subsubsection{論理合成,FPGAでの回路実現}
ここに写真から読み取った情報を書く

\subsection{考察}
