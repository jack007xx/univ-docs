\section{実験4-1}
\subsection{実験の目的,概要}
本実験では、ディスプレイに61種類の文字を順番に表示させるプログラムを、クロスコンパイルし、実験2で作成したプロセッサ上で動作させる。
その際,事前にプロセッサの動作について予想しておき,結果と比較する。

これによって、プログラムの作成からFPGA上での実現までの全体のフローを確認することを目的とする。

\subsection{実験方法}
\subsubsection{クロスコンパイル,メモリイメージファイルの作成}
以下の,ディスプレイに61種類の文字を表示するプログラム,print\_all\_char.cを配置した。
\lstinputlisting[caption=print\_all\_char.c,label=printallchar.c4-1]{src/obj4-1/print_all_char.c}

以下のコマンドで,クロスコンパイルを行い,MIPSマシンコードprint\_all\_char.binを得た。
\begin{lstlisting}[caption={クロスコンパイル,メモリイメージファイルの作成},label={クロスコンパイル,イメージファイルの作成4-1}]
  $ cross_compile.sh print_all_char.c
  $ bin2v print_all_char.bin
\end{lstlisting}

\subsubsection{命令列の確認}
生成された,rom8x1024.mifを確認して,以下の点について結果を予測する。
\begin{itemize}
  \item 最初にPC=0x0074の命令を実行したときのレジスタ2番めの値
  \item 最初にPC=0x0078の命令を実行したときのPCの値
\end{itemize}

\subsubsection{論理合成,FPGAでの回路実現}
以下のコマンドで,論理合成,ダウンロードを行なった。
\begin{lstlisting}[caption={論理合成,ダウンロード},label={論理合成,ダウンロード4-1}]
  $ mv rom8x1024.mif mips_de10-lite/
  $ cd mips_de10-lite/
  $ quartus_sh --flow compile MIPS_Default
  $ quartus_pgm MIPS_Default.cdf
\end{lstlisting}

FPGA上の動作を確認した。

\subsection{実験結果}
\subsubsection{クロスコンパイル,メモリイメージファイルの作成}
以下のような,メモリイメージファイルが生成された。
\lstinputlisting[caption=rom8x1024.mif,label=rom8x1024.mif4-1]{src/obj4-1/mips_de10-lite/rom8x1024.mif}

\subsubsection{命令列の確認}
ソースコード\ref{rom8x1024.mif4-1}を確認し,以下のような予想を立てた。
\begin{itemize}
  \item 最初にPC=0x0074の命令を実行したときのレジスタ2番目の値
  \begin{itemize}
    \item 1
  \end{itemize}
  \item 最初にPC=0x0078の命令を実行したときのPCの値
  \begin{itemize}
    \item PC=0x0038
  \end{itemize}
\end{itemize}

\subsubsection{論理合成,FPGAでの回路実現}
FPGA上で予想した点について確認したが,レジスタの値は0で,PCの値も更新されずに通常通り+4されて進んだ。

\subsection{考察}
ソースコード\ref{printallchar.c4-1}は,文字コードとして用意した値をインクリメントして,文字コードの並び順に61種類の文字列を表示するプログラムである。

そこから生成されたソースコード\ref{rom8x1024.mif4-1}を確認して行なった予想では,以下のような考察を行った。
\begin{itemize}
  \item 最初にPC=0x0074の命令を実行したときのレジスタ2番目の値
  \begin{itemize}
    \item 1
    \item 初回実行時,PC=0x0030から0x006cにジャンプしてくる
    \item 0x006cでは,RAM[REG[30]]の値がREG[2]に代入される
    \item RAM[REG[30]]は,0x002cで0に初期化されているのでのでREG[2]=0
    \item 0x0074では,REG[2]が61より小さければ,1を代入するので,REG[2]=1
  \end{itemize}
  \item 最初にPC=0x0078の命令を実行したときのPCの値
  \begin{itemize}
    \item PC=0x0038
    \item 先の予想より,REG[2]=1
    \item このとき,PCの値はPC=PC+$(65520*4)_{10}$
    \item 16進に直して加算すると,オーバーフローしてPC=0x0038
  \end{itemize}
\end{itemize}

FPGA上の動作は,予想と異なった。
これは,プロセッサにsltiu命令と,bne命令が実装されていないからであると考えられる。
オペコードを読んでも,定義されていないので何もせずにPCが進んだ。
