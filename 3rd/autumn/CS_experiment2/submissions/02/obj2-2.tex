\section{実験2-2}
\subsection{実験の目的,概要}

\subsection{実験方法}
\subsubsection{j命令のモジュール追加設計}
以下の点について,cpu.vを編集して,新たにjp\_selモジュールを定義した。
\begin{lstlisting}[caption={addiu命令の追加設計},label={addiu命令の追加設計}]
// ワイヤ宣言
wire  [31:0]     jp_sel_d0;  // jp 選択回路モジュール データ 1
wire  [31:0]     jp_sel_d1;  // jp 選択回路モジュール データ 2
wire  [31:0]     jp_sel_s;  // jp 選択回路モジュール セレクト信号
wire  [31:0]     jp_sel_y;  // jp 選択回路モジュール 出力

// セレクタの実体化
mux32_32_32  jp_sel(jp_sel_d0, jp_sel_d1, jp_sel_s, jp_sel_y);
mux32_32_32  pc_sel(pc_sel_d0, pc_sel_d1, pc_sel_s, pc_sel_y);

// pc_nextへの接続変更
assign pc_next = jp_sel_y;

// 結線
assign jp_sel_d0 = pc_sel_y;
assign jp_sel_d1 = sh_j_y;
assign jp_sel_s = jp;
\end{lstlisting}

\subsubsection{j命令のメイン制御回路追加設計}
以下の点について,main\_ctrl.vを編集した。
\begin{lstlisting}[caption={sw命令の追加設計},label={sw命令の追加設計}]
// 命令コードの定義を追加
`define  	 J  6'b000010

// jp_selへのセレクト信号を追加
assign  jp = ((op_code == `J) || ((op_code == `JAL) && 00)) ? 1'b1 : 1'b0;

// write_enable制御信号の追加
`J:      reg_write_enable_tmp = 1'b0;
\end{lstlisting}

\subsubsection{論理合成,FPGAでの回路実現}
以下のように,コンパイルを行ない,dmesgで接続を確認した。
その後FPGAにダウンロードして,FPGAでの動作を確認した。

\begin{lstlisting}[caption={コンパイル,ダウンロード},label={コンパイル,ダウンロード2-2}]
  $ cp rom8x1024.mif ./mipsde_10-lite
  $ cd ./mipsde_10-lite
  $ quartus_sh --flow compile MIPS_Default
  $ quartus_pgm MIPS_Default.cdf 
\end{lstlisting}

\subsection{実験結果}
\subsubsection{メモリイメージファイルの作成}

\subsubsection{命令列の確認}

\subsubsection{論理合成,FPGAでの回路実現}

\subsection{考察}
