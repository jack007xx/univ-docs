\documentclass[a4paper,15pt]{jsarticle}

% 数式
\usepackage{amsmath,amsfonts}
\usepackage{bm}
% 画像
\usepackage[dvipdfmx]{graphicx}

\usepackage{here}
\usepackage{url}

\usepackage{listingsutf8,jlisting} %日本語のコメントアウトをする場合jlistingが必要
%ここからソースコードの表示に関する設定
\lstset{
  basicstyle={\ttfamily},
  identifierstyle={\small},
  commentstyle={\smallitshape},
  keywordstyle={\small\bfseries},
  ndkeywordstyle={\small},
  stringstyle={\small\ttfamily},
  frame={tb},
  breaklines=true,
  columns=[l]{fullflexible},
  numbers=left,
  xrightmargin=0zw,
  xleftmargin=3zw,
  numberstyle={\scriptsize},
  stepnumber=1,
  numbersep=1zw,
  lineskip=-0.5ex
}

\begin{document}

\title{コンピュータ科学実験レポート}
\author{坪井正太郎(101830245)}
\date{\today}
\maketitle

\section*{はじめに}

\section*{各実験}


\begin{thebibliography}{99}
  \bibitem{ALTERA}AN 584: 高度な FPGA デザインにおける  タイミング・クロージャ手法, ALTERA,2009,\url{https://www.intel.co.jp/content/dam/altera-www/global/ja_JP/pdfs/literature/an/an584_j.pdf}
  \bibitem{RTL} RTLとゲートレベルを混在させた最適な論理回路設計に関する研究,ZHANG, ZHIFEI,2014,\url{https://dspace.jaist.ac.jp/dspace/handle/10119/12013}
\end{thebibliography}
すべて2020年10月22日閲覧

\end{document}
