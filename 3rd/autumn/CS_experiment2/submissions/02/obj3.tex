\section{実験3}
\subsection{実験の目的、概要}
本実験では、print\_B.binにコンパイルされるC言語の記述を、print\_B\_while.binを生成するように書き換える。
また、作成したコードは実際にコンパイルし、正しいデータと比較する。

この実験で、FPGA上で文字を表示するためのプログラムの書き方、MIPS向けクロスコンパイルの方法を確認することを目的とする。

\subsection{実験方法}
print\_B.cを編集して、以下のようなファイルmy\_print\_B\_while.cを作成した。
\lstinputlisting[caption=my\_print\_B\_while.c,label=myprintBwhile.c]{./src/obj3/my_print_B_while.c}

以下のコマンドで、MIPS用にクロスコンパイルして、生成されたバイナリからメモリイメージファイルを作成した。

\begin{lstlisting}[caption={クロスコンパイル},label={クロスコンパイル}]
  $ cross_compile.sh my_print_B_while.c
  $ bin2v my_print_B_while.bin
\end{lstlisting}

\subsection{実験結果}
コンパイルの結果、このようなメモリイメージファイルが生成された。
\lstinputlisting[caption=rom8x1024.mif,label=rom8x1024.mif3]{src/obj3/rom8x1024.mif}

内容は、実験2-1で使用したソースコード\ref{rom8x1024.mif2-1}と同じだった。

\subsection{考察}
編集したCのコードには、繰り返し文字を入力させるためのwhile文を追記した。
実験1-2で考察したようにメモリマップドIOによって入出力が行われている。
このとき、コンパイラによる最適化で情報が潰されないように、volatileキーワードで最適化を抑制している。

このプログラムから、FPGA上で文字を出力させるプログラムの記述と、クロスコンパイルの方法を確認することができた。

クロスコンパイルによって、ターゲットマシンが低速で、コンパイルが不可能、時間がかかる場合でも、異なるアーキテクチャのマシンで実行ファイルが生成できることがわかる。
