\section{実験4-2}
\subsection{実験の目的、概要}
実験4-1では、プロセッサにsltiu、bne、lw命令が実装されていないため、プログラムが正しく動作しなかった。
本実験では、3つの命令をプロセッサに実装し、4-1と同じプログラムを動作させて比較する。

これによって、ループ処理に関する命令実装について確認する。

\subsection{実験方法}
\subsubsection{sltiu命令のメイン制御回路追加設計}
以下の点について、main\_ctrl.vを編集した。
\begin{lstlisting}[caption={sltiu命令の追加設計},label={sltiu命令の追加設計}]
// 命令コードの定義を追加
`define   SLTIU 6'b001011

// is_branch制御信号の追加
`SLTIU:  is_branch_ctrl_tmp = 3'b110;  // do nothing

// ALU の入力ポート B へ流すデータを選択するセレクト信号の記述
`SLTIU:  alu_b_sel1_s_tmp = 1'b1;

// ALU演算の制御信号の追加
`SLTIU:  alu_op_tmp = 3'b111;

// write_enable制御信号の追加
`SLTIU:  reg_write_enable_tmp = 1'b1;

// ALU_RAM_SELの制御信号の追加
`SLTIU:  alu_ram_sel_s_tmp = 1'b0;

// rt rd のセレクト信号の追加
`SLTIU:  reg_widx_sel1_s_tmp = 1'b0;

// PCとALUのセレクト信号の追加
`SLTIU:  link_tmp = 1'b0;
\end{lstlisting}

\subsubsection{bne命令のメイン制御回路追加設計}
以下の点について、main\_ctrl.vを編集した。
\begin{lstlisting}[caption={bne命令の追加設計},label={bne命令の追加設計}]
  // 命令コードの定義を追加
  `define    BNE  6'b000101

  // is_branch制御信号の追加
  `BNE:    is_branch_ctrl_tmp = 3'b001;  // !=, NEQ

  // レジスタと即値のセレクタ信号の追加
  `BNE:    alu_b_sel1_s_tmp = 1'b0;

  // 符号拡張モジュール sign_ext の制御信号
  || (op_code == `BNE)                       

  // write_enable制御信号の追加
  `BNE:    reg_write_enable_tmp = 1'b0;
\end{lstlisting}

\subsubsection{lw命令のメイン制御回路追加設計}
以下の点について、main\_ctrl.vを編集した。
\begin{lstlisting}[caption={lw命令の追加設計},label={lw命令の追加設計}]
  // 命令コードの定義を追加
  `define     LW  6'b100011

  // is_branch制御信号の追加
  `LW:     is_branch_ctrl_tmp = 3'b110;  // do nothing
  
  // レジスタと即値のセレクタ信号の追加
  `LW:     alu_b_sel1_s_tmp = 1'b1;

  // ALU演算の制御信号の追加
  `LW:     alu_op_tmp = 3'b000;

  // write_enable制御信号の追加
  `LW:     reg_write_enable_tmp = 1'b1;

  // 演算結果とRAMの値のセレクト信号の追加
  `LW:     alu_ram_sel_s_tmp = 1'b1;

  // rt rd のセレクト信号の追加
  `LW:     reg_widx_sel1_s_tmp = 1'b0;

  // レジスタに書き込む、ALUの出力orRAMのデータと、PCの値のセレクタ信号の追加
  `LW:     link_tmp = 1'b0;
\end{lstlisting}

\subsubsection{論理合成、FPGAでの回路実現}
以下のように、コンパイルを行ない、FPGAにダウンロードして、FPGAでの動作を確認した。

\begin{lstlisting}[caption={コンパイル、ダウンロード},label={コンパイル、ダウンロード4-2}]
  $ cp rom8x1024.mif ./mipsde_10-lite
  $ cd ./mipsde_10-lite
  $ quartus_sh --flow compile MIPS_Default
  $ quartus_pgm MIPS_Default.cdf 
\end{lstlisting}  

\subsection{実験結果}
\subsubsection{論理合成、FPGAでの回路実現}
ダウンロードして実行した。
200Hzでカウントを進めたところ、画面上に61種類の文字が表示された。

実験4-1の予想通りの値がレジスタ、メモリに確認できた。

\subsection{考察}
条件つきループを実行するために、足りない命令を実装した。
条件を判定するためにsltiu命令を、判定に基づいて分岐するためにbne命令を、ループ内での状態を判定するためにlw命令を実装した。
