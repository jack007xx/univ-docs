\section{実験6-2}
\subsection{実験の目的、概要}
実験5-2で作成したプロセッサにjr命令が足りないことが、実験6-1で確認できた。
本実験では、プロセッサにjr命令を追加実装し、動作を確認する。
その際、実験6-1での予想と実際を比較する。

これによって、関数からの復帰時のプロセッサの動作、レジスタから読み取れるデータなどを確認することを目的とする。

\subsection{実験方法}

\subsection{追加設計}
cpu.vに、以下の変更を加え、jpr\_selモジュールを追加した。
\begin{lstlisting}[caption={cpu.vの追加設計},label={cpu.vの追加設計6-2}]
jpr_selの入出力ワイヤを定義
wire  [31:0]     jpr_sel_d0;  // jpr 選択回路モジュール データ 1
wire  [31:0]     jpr_sel_d1;  // jpr 選択回路モジュール データ 2
wire  [31:0]     jpr_sel_s;  // jpr 選択回路モジュール セレクト信号
wire  [31:0]     jpr_sel_y;  // jpr 選択回路モジュール 出力

32bitマルチプレクサモジュールをjpr_selとして宣言する
入出力はさっき定義したワイヤを使う
mux32_32_32  jpr_sel(jpr_sel_d0, jpr_sel_d1, jpr_sel_s, jpr_sel_y);

セレクタの出力をPCに接続する
assign pc_next = jpr_sel_y;
代わりに,割り当てられていたjp_sel_yをコメントアウトする
//assign pc_next = jp_sel_y;

セレクタの入力を割り当てる
assign jpr_sel_d0 = jp_sel_y;
assign jpr_sel_d1 = alu_ram_sel_y;
assign jpr_sel_s = jpr;
\end{lstlisting}

main\_ctrl.vに以下の変更を加え、jr命令が実行されるときの制御信号を定義した。
\begin{lstlisting}[caption={main\_ctrl.vの追加設計},label={mainctrl.vの追加設計6-2}]
jpr信号を出力する条件として、jrのファンクションコードを追加する。
|| Rfunc == 6'b001000)) ? 1'b1 : 1'b0;

レジスタの書き込み制御信号の出力条件を追加
if (Rfunc == 6'b001000) begin
  reg_write_enable_tmp = 1'b0;
end else begin
  reg_write_enable_tmp = 1'b1;
end
\end{lstlisting}
以下の操作で、論理合成し、FPGAにダウンロードした。
\begin{lstlisting}[caption={論理合成、ダウンロード},label={論理合成、ダウンロード6-2}]
$  quartus_sh --flow compile MIPS_Default
$  quartus_pgm MIPS_Default.cdf
\end{lstlisting}

実験6-1で予想した点と、キーボードからの入力に対する反応について確認した。

\subsection{実験結果}
予想した点について結果は、以下のようになった。
\begin{itemize}
  \item 最初にPC=0x0074を実行した直後のREG[31]の値
        \begin{itemize}
          \item REGWRITED=0x00400080,IDX=0x1f,WEN=1となり、REG[31]にPCの値が退避された
        \end{itemize}
  \item 最初にPC=0x0804を実行した直後の,PCの値
        \begin{itemize}
          \item PC=0x00400080と、予想通りの結果になった
        \end{itemize}
\end{itemize}

画面上には、"HELLO!!"という文字列が表示された。
キーボードとの入出力の結果、以下のように、キーボードで入力した文字列がそのまま出力された。
("STRING="の後の文字列がキーボードから入力した文字列である。)
\begin{lstlisting}[caption={実行結果},label={実行結果6-2}]
HELLO!!
STRING=HELLO
ECHO HELLO
STRING=BYE
ECHO BYE
\end{lstlisting}

\subsection{考察}
予想した点のjr命令は、最初のjal命令で退避したPCの値を復元して、PC=0x00400080にジャンプした。
jr命令を実装したことによって、関数から戻って手続きをすすめることができた。
これにより、画面への出力と、キーボードからの入力を受け取る関数を正しく実行することができた。

この実験で、関数からの復帰時のプロセッサの動作を確認することができた。
