\section{実験6-2}
\subsection{実験の目的、概要}
実験5-2で作成したプロセッサにjr命令が足りないことが、実験6-1で確認できた。
本実験では、プロセッサにjr命令を追加実装し、動作を確認する。
その際、実験6-1での予想と実際を比較する。

これによって、関数からの復帰時のプロセッサの動作、レジスタから読み取れるデータなどを確認することを目的とする。

\subsection{実験方法}

\subsection{追加設計}
cpu.vに、以下の変更を加え、jpr\_selモジュールを追加した。
\begin{lstlisting}[caption={cpu.vの追加設計},label={cpu.vの追加設計6-2}]
jpr_selの入出力ワイヤを定義
wire  [31:0]     jpr_sel_d0;  // jpr 選択回路モジュール データ 1
wire  [31:0]     jpr_sel_d1;  // jpr 選択回路モジュール データ 2
wire  [31:0]     jpr_sel_s;  // jpr 選択回路モジュール セレクト信号
wire  [31:0]     jpr_sel_y;  // jpr 選択回路モジュール 出力

32bitマルチプレクサモジュールをjpr_selとして宣言する
入出力はさっき定義したワイヤを使う
mux32_32_32  jpr_sel(jpr_sel_d0, jpr_sel_d1, jpr_sel_s, jpr_sel_y);

セレクタの出力をPCに接続する
assign pc_next = jpr_sel_y;
代わりに,割り当てられていたjp_sel_yをコメントアウトする
//assign pc_next = jp_sel_y;

セレクタの入力を割り当てる
assign jpr_sel_d0 = jp_sel_y;
assign jpr_sel_d1 = alu_ram_sel_y;
assign jpr_sel_s = jpr;
\end{lstlisting}

main\_ctrl.vに以下の変更を加え、jr命令が実行されるときの制御信号を定義した。
\begin{lstlisting}[caption={main\_ctrl.vの追加設計},label={mainctrl.vの追加設計6-2}]
jpr信号を出力する条件として、jrのファンクションコードを追加する。
|| Rfunc == 6'b001000)) ? 1'b1 : 1'b0;

レジスタの書き込み制御信号の出力条件を追加
if (Rfunc == 6'b001000) begin
  reg_write_enable_tmp = 1'b0;
end else begin
  reg_write_enable_tmp = 1'b1;
end
\end{lstlisting}

\subsection{実験結果}

\subsection{考察}
