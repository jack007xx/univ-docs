\section{実験7}
\subsection{実験の目的、概要}
本実験では、実験6-2で作成したプロセッサ上で、キーボード入力に応答を行うプログラムを動作させる。
このプログラムは、プロセッサに実装されていない命令を使用するため、正常に動作しない。

本実験では、このプログラムを、正常に動作させることを目的とする。

\subsection{実験方法1}
以下のようなプログラムを配置した。
\lstinputlisting[caption=sosuu.c,label=sosuu.c7]{src/07/sosuu.c}

\subsubsection{クロスコンパイル、メモリイメージファイルの作成}
以下の操作でクロスコンパイルし、メモリイメージファイルを作成した。
\begin{lstlisting}[caption={クロスコンパイル、メモリイメージファイルの作成},label={クロスコンパイル、メモリイメージファイルの作成5-1}]
$  cross_compile.sh sosuu.c
$  bin2v sosuu.bin
\end{lstlisting}

\subsubsection{命令列の確認、動作予想}
生成された、rom8x1024.mifを確認して、以下の点について結果を予測した。
\begin{itemize}
  \item 命令メモリの0x082の命令はどのような命令か
  \item 命令メモリの0x082の命令はsosuu\_check()のどの記述に対応するか
\end{itemize}

\subsubsection{論理合成、ダウンロード}
以下の操作で、論理合成し、FPGAにダウンロードした。
\begin{lstlisting}[caption={論理合成、ダウンロード},label={論理合成、ダウンロード6-1}]
$  cp rom8x1024.mif ./mips_de10-lite/
$  cd ./mips_de10-lite/
$  quartus_sh --flow compile MIPS_Default
$  quartus_pgm MIPS_Default.cdf
\end{lstlisting}

FPGA上で、HELLO, NUM=と表示されたら、"20"を入力し、その結果を確認した。

このプログラムは正しく動作しない。
この問題について、解決する方法を2つ考えた。

\subsection{実験結果1}
\subsubsection{命令列からの予想}
メモリイメージファイルから、以下のような予想をたてた。
\begin{itemize}
  \item 命令メモリの0x082の命令はどのような命令か
        \begin{itemize}
          \item 2つのレジスタの内容を符号なし整数と解釈して除算し、商はLO、余りはHIに格納する、divu命令
          \item rsに3,rtに2を指定しているため、LO=REG[3]/REG[2]、LO=REG[3]\%REG[2]
        \end{itemize}
  \item 命令メモリの0x082の命令はsosuu\_check()のどの記述に対応するか
        \begin{itemize}
          \item ソースコード\ref{sosuu.c7}の88行目 if ((kouho \% tester) == 0)に対応する
          \item RAMに0(false)を代入しており、変数をlwして++2していることから分かる
        \end{itemize}
\end{itemize}

\subsubsection{プログラムの実行}
以下のような表示となり、素数が表示されず、正しく実行されなかった。
\begin{lstlisting}[caption={sosuu.c実行結果1},label={sosuu.c実行結果17}]
HELLO
NUM=20
ECHO 20
03 05 08
NUM=
\end{lstlisting}

この問題について、以下のような解決方法を考えた。
\begin{itemize}
  \item プロセッサが実行できるようにプログラムを変更
  \item プロセッサに命令を足りない命令を追加
\end{itemize}

\subsection{実験方法2}
考案した解決方法のうち、今回は、「プロセッサが実行できるようにプログラムを変更」という方法で解決した。

\subsubsection{Cプログラムの変更}
sosuu.cのプログラムを以下のように変更した。
\lstinputlisting[caption=sosuu.c,label=sosuu.c27]{src/07/sosuu’.c}

\subsubsection{論理合成、ダウンロード}
以下の操作で、論理合成し、FPGAにダウンロードした。
\begin{lstlisting}[caption={論理合成、ダウンロード},label={論理合成、ダウンロード6-1}]
$  cp rom8x1024.mif ./mips_de10-lite/
$  cd ./mips_de10-lite/
$  quartus_sh --flow compile MIPS_Default
$  quartus_pgm MIPS_Default.cdf
\end{lstlisting}

FPGA上で実行し、プログラムの動作を確認した。

\subsubsection{実験結果2}
画面上には、以下のような結果が表示され、正しく素数を判定できていた。
\begin{lstlisting}[caption={実行結果2},label={実行結果27}]
HELLO
NUM=20
ECHO 20
03 05 07 11 13 17 19
NUM=
\end{lstlisting}

\subsection{考察}
このプロセッサには、divu命令が実装されていない。
そのため、正しくプログラムを実行することができなかったが、プログラムに変更することで、正しく動作させることに成功した。

必要な変更は、剰余を用いずにkouhoをtesterで割り切ることができるか判定することであった。
ここでは、testerにtesterを加えていき、kouhoを超える前に、kouho==testerとなるかどうかを判定した。
これによって、kouhoがtesterで割り切ることができるか判定できると考えた。
実際に、判定プログラムは正しく動作した。
