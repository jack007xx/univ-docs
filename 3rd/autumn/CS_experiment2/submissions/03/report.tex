% \documentclass[a4j,15pt,draft]{jsarticle}
\documentclass[a4j,15pt]{jsarticle}

% 数式
\usepackage{amsmath,amsfonts}
\usepackage{bm}
% 画像
\usepackage[dvipdfmx]{graphicx}

\usepackage{here}
\usepackage{url}

\usepackage{listingsutf8,jlisting} %日本語のコメントアウトをする場合jlistingが必要
%ここからソースコードの表示に関する設定
\lstset{
  basicstyle={\ttfamily},
  identifierstyle={\small},
  commentstyle={\smallitshape},
  keywordstyle={\small\bfseries},
  ndkeywordstyle={\small},
  stringstyle={\small\ttfamily},
  frame={tb},
  breaklines=true,
  columns=[l]{fullflexible},
  numbers=left,
  xrightmargin=0zw,
  xleftmargin=3zw,
  numberstyle={\scriptsize},
  stepnumber=1,
  numbersep=1zw,
  lineskip=-0.5ex
}

\begin{document}

\title{コンピュータ科学実験レポート}
\author{坪井正太郎(101830245)}
\date{\today}
\maketitle

\section*{はじめに}
この実験では、一部の命令が実装されていないプロセッサに、適切な命令を実装して、条件付きループ命令を含む動作を行えるようにする。

また、各実験では,シュミレータや論理合成のソフトウェアを使うために,以下の設定を行う。
端末を終了した場合,再度sourceコマンドを実行する。
\begin{lstlisting}[caption={設定の読み込み},label={設定の読み込み}]
  $ ln -s /pub1/jikken/eda3/cadsetup.bash.altera ~/
  $ source ~/cadsetup.bash.altera
\end{lstlisting}

HDLのコンパイルにはQuartus Primeを、機能レベルシュミレーションにはModel Simを使用した。

バイナリファイルの内容は、hexdumpコマンドによる。
一番左のカラムは、hexdumpの行数である。

\section*{各実験}
\section{実験5-1}
\subsection{実験の目的、概要}
この実験では、実験4-2で作成したプロセッサで画面上に文字を表示するCプログラムを実行する
その中で関数呼び出しを行い、動作を予想し、結果を確認する。

これによって、現在足りていない機能を確認することを目的とする。

\subsection{実験方法}
以下のプログラムを配置した。
\lstinputlisting[caption=my\_print.c,label=myprint.c5-1]{src/05/my_print.c}

\subsubsection{クロスコンパイル、メモリイメージファイルの作成}
以下の操作でクロスコンパイルし、メモリイメージファイルを作成した。
\begin{lstlisting}[caption={クロスコンパイル、メモリイメージファイルの作成},label={クロスコンパイル、メモリイメージファイルの作成5-1}]
$  cross_compile.sh my_print.c
$  bin2v my_print.bin
\end{lstlisting}

\subsubsection{命令列の確認、動作予想}
生成された、rom8x1024.mifを確認して、以下の点について結果を予測した。
\begin{itemize}
  \item 最初にPC=0x0074を実行した直後のREG[31]の値
  \item 最初にPC=0x0074を実行した直後のPCの値
\end{itemize}

\subsubsection{論理合成、ダウンロード}
以下の操作で、論理合成し、FPGAにダウンロードした。
\begin{lstlisting}[caption={論理合成、ダウンロード},label={論理合成、ダウンロード5-1}]
$  cp rom8x1024.mif ./mips_de10-lite/
$  cd ./mips_de10-lite/
$  quartus_sh --flow compile MIPS_Default
$  quartus_pgm MIPS_Default.cdf
\end{lstlisting}

クロックを手動モードで送り、70個ほどの命令を実行、予想した点と、ディスプレイに表示されるはずの文字について確認した。

\subsection{実験結果}
\subsubsection{命令列の確認、動作予想}
メモリイメージファイルが生成された。
% \lstinputlisting[caption=rom1024.mif,label=rom1024.mif]{src/05/rom8x1024.mif}

命令列を確認して、このような予想をたてた。
\begin{itemize}
  \item 最初にPC=0x0074を実行した直後のREG[31]の値
  \begin{itemize}
    \item REG[31]=0x0078
  \end{itemize}
  \item 最初にPC=0x0074を実行した直後のPCの値
  \begin{itemize}
    \item PC=0x00a0
  \end{itemize}
\end{itemize}

\subsubsection{FPGAでの実行結果}
予想した点について結果は、以下のようになった。
\begin{itemize}
  \item 最初にPC=0x0074を実行した直後のREG[31]の値
  \begin{itemize}
    \item REGWRITED=00000000,WEN=0であり、レジスタへの書き込みは発生していない
  \end{itemize}
  \item 最初にPC=0x0074を実行した直後のPCの値
  \begin{itemize}
    \item PC=0x0078であり、ジャンプはしていない
  \end{itemize}
\end{itemize}

ディスプレイに文字は表示されなかった。

\subsection{考察}
このプロセッサには、jal命令が実装されていないので、プログラムが正しく動作しなかった。
jal命令が正しく実装されている場合、31番目のレジスタに次の戻り先プログラムカウンタの位置が退避されて、プログラムカウンタの位置が命令で指定された場所に更新されるはずである。

\section{実験5-2}
\subsection{実験の目的、概要}
実験4-2で作成したプロセッサにjal命令が足りないことが、実験5-1で確認できた。
本実験では、プロセッサにjal命令を追加実装し、動作を確認する。
その際、実験5-1での予想と実際を比較する。

これによって、関数呼び出し時のプロセッサの動作、レジスタに保存されるデータなどを確認することを目的とする。

\subsection{実験方法}
\subsection{追加設計}
main\_ctrl.vに以下の変更を加え、jal命令のオペコードと、実行されるときの制御信号を定義した。
\begin{lstlisting}[caption={main\_ctrl.vの追加設計},label={mainctrl.vの追加設計5-2}]
オペコードの定義
`define    JAL  6'b000011  //  jump and link (J 形式)

jp_sel モジュールへの制御信号の記述
+=4の値ではなく、命令文中のアドレスを選択する
assign  jp = ((op_code == `J) || (op_code == `JAL)) ? 1'b1 : 1'b0;

レジスタのwrite_enable制御信号の追加
`JAL:    reg_write_enable_tmp = 1'b1;

レジスタに流すデータのセレクト信号の追加
REG[31]に次のPC値が保存されるようになる
`JAL:    link_tmp = 1'b1;
\end{lstlisting}

\subsubsection{論理合成、ダウンロード}
以下の操作で、論理合成し、FPGAにダウンロードした。
\begin{lstlisting}[caption={論理合成、ダウンロード},label={論理合成、ダウンロード5-2}]
$  cp rom8x1024.mif ./mips_de10-lite/
$  cd ./mips_de10-lite/
$  quartus_sh --flow compile MIPS_Default
$  quartus_pgm MIPS_Default.cdf
\end{lstlisting}

実験5-1で予想した点と、ディスプレイに表示されるはずの文字について確認した。

\subsection{実験結果}
予想した点について結果は、以下のようになった。
\begin{itemize}
  \item 最初にPC=0x0074を実行した直後のREG[31]の値
        \begin{itemize}
          \item REGWRITED=00400078,index=0x1f(=31),write\_enable=1が読み取れた
          \item 予想通り、次に実行するPCの値が、REG[31]
        \end{itemize}
  \item 最初にPC=0x0074を実行した直後のPCの値
        \begin{itemize}
          \item 予想通りPC=0x=0x00a0であり、ジャンプしている
        \end{itemize}
\end{itemize}

画面上には、"HELLO!!"という文字列が表示された。

\subsection{考察}
jal命令を実装したことで、\$raにPCの値が退避され、即値で指定した値にジャンプできるようになった。
そのため、予想した点について正しい動作を確認することができた。
これは、my\_print関数の呼び出しに成功しているということであると考えられる。
実際に、画面上に文字列が表示されることが分かった。

この実験で、関数呼び出し時のプロセッサの動作、レジスタに保存されるデータを確認する事ができた。

\section{実験6-1}
\subsection{実験の目的、概要}
実験5-2でjal命令を追加したプロセッサ上で、キーボードからの入力を受け取るCプログラムを実行する。
その中で関数からの復帰を行い、それらの動作を予想し、結果を確認する。

これによって、現在足りていない機能を確認することを目的とする。

\subsection{実験方法}
以下のプログラムを配置した。
\lstinputlisting[caption=my\_scan.c,label=myscan.c6-1]{src/06/my_scan.c}

\subsubsection{クロスコンパイル、メモリイメージファイルの作成}
以下の操作でクロスコンパイルし、メモリイメージファイルを作成した。
\begin{lstlisting}[caption={クロスコンパイル、メモリイメージファイルの作成},label={クロスコンパイル、メモリイメージファイルの作成5-1}]
$  cross_compile.sh my_scan.c
$  bin2v my_scan.bin
\end{lstlisting}

\subsubsection{命令列の確認、動作予想}
生成された、rom8x1024.mifを確認して、以下の点について結果を予測した。
\begin{itemize}
  \item 最初にPC=0x007cを実行したときの,REG[31]の値
  \item 最初にPC=0x0804を実行した直後の,PCの値
\end{itemize}

\subsubsection{論理合成、ダウンロード}
以下の操作で、論理合成し、FPGAにダウンロードした。
\begin{lstlisting}[caption={論理合成、ダウンロード},label={論理合成、ダウンロード6-1}]
$  cp rom8x1024.mif ./mips_de10-lite/
$  cd ./mips_de10-lite/
$  quartus_sh --flow compile MIPS_Default
$  quartus_pgm MIPS_Default.cdf
\end{lstlisting}

クロックを手動モードで送り、70個ほどの命令を実行、予想した点と、ディスプレイに表示されるはずの文字について確認した。

\subsection{実験結果}
\subsubsection{命令列の確認、動作予想}
メモリイメージファイルが生成された。
% \lstinputlisting[caption=rom1024.mif,label=rom1024.mif]{src/06/rom8x1024.mif}

生成された命令列を確認して、このような予想をたてた。
\begin{itemize}
  \item 最初にPC=0x007cを実行したときの,REG[31]の値
        \begin{itemize}
          \item 
        \end{itemize}
  \item 最初にPC=0x0804を実行した直後の,PCの値
\end{itemize}

\subsubsection{FPGAでの実行結果}
予想した点について結果は、以下のようになった。
\begin{itemize}
  \item 最初にPC=0x0074を実行した直後のREG[31]の値
        \begin{itemize}
          \item REGWRITED=0x00400080,IDX=0x1f,WEN=1となり、\$raにPCの値が退避された
        \end{itemize}
  \item 最初にPC=0x0804を実行した直後の,PCの値
        \begin{itemize}
          \item PC=0x00400808
        \end{itemize}
\end{itemize}

画面上には、"HELLO!!"のみ表示された。

\subsection{考察}
予想した点について、PC=0x0074の命令は正しく動作しているが、PC=0x0804の命令については正しく実行できていないということが分かる。
これは、プロセッサにjr命令が実装されていないためであると考えられる。

このプロセッサには、jal命令は実装されているが、jr命令が実装されていないので、関数に入ることはできても、関数から戻ることができない。
そのため、プログラム中のmy\_print関数の実行はできるが、戻ることができないため、次の手続きに進むことができないと考えられる。

\section{実験6-2}
\subsection{実験の目的、概要}
実験5-2で作成したプロセッサにjr命令が足りないことが、実験6-1で確認できた。
本実験では、プロセッサにjr命令を追加実装し、動作を確認する。
その際、実験6-1での予想と実際を比較する。

これによって、関数からの復帰時のプロセッサの動作、レジスタから読み取れるデータなどを確認することを目的とする。

\subsection{実験方法}

\subsection{追加設計}
cpu.vに、以下の変更を加え、jpr\_selモジュールを追加した。
\begin{lstlisting}[caption={cpu.vの追加設計},label={cpu.vの追加設計6-2}]
jpr_selの入出力ワイヤを定義
wire  [31:0]     jpr_sel_d0;  // jpr 選択回路モジュール データ 1
wire  [31:0]     jpr_sel_d1;  // jpr 選択回路モジュール データ 2
wire  [31:0]     jpr_sel_s;  // jpr 選択回路モジュール セレクト信号
wire  [31:0]     jpr_sel_y;  // jpr 選択回路モジュール 出力

32bitマルチプレクサモジュールをjpr_selとして宣言する
入出力はさっき定義したワイヤを使う
mux32_32_32  jpr_sel(jpr_sel_d0, jpr_sel_d1, jpr_sel_s, jpr_sel_y);

セレクタの出力をPCに接続する
assign pc_next = jpr_sel_y;
代わりに,割り当てられていたjp_sel_yをコメントアウトする
//assign pc_next = jp_sel_y;

セレクタの入力を割り当てる
assign jpr_sel_d0 = jp_sel_y;
assign jpr_sel_d1 = alu_ram_sel_y;
assign jpr_sel_s = jpr;
\end{lstlisting}

main\_ctrl.vに以下の変更を加え、jr命令が実行されるときの制御信号を定義した。
\begin{lstlisting}[caption={main\_ctrl.vの追加設計},label={mainctrl.vの追加設計6-2}]
jpr信号を出力する条件として、jrのファンクションコードを追加する。
|| Rfunc == 6'b001000)) ? 1'b1 : 1'b0;

レジスタの書き込み制御信号の出力条件を追加
if (Rfunc == 6'b001000) begin
  reg_write_enable_tmp = 1'b0;
end else begin
  reg_write_enable_tmp = 1'b1;
end
\end{lstlisting}

\subsection{実験結果}

\subsection{考察}

\section{実験7}
\subsection{実験の目的、概要}
本実験では、実験6-2で作成したプロセッサ上で、キーボード入力に応答を行うプログラムを動作させる。
このプログラムは、プロセッサに実装されていない命令を使用するため、正常に動作しない。
本実験では、このプログラムを、正常に動作させることを目標とする。

これによって、プロセッサで足りない機能などの問題点から解決法を考え、実装することができるようになること、を目的とする。

\subsection{実験方法}

\subsection{実験結果}

\subsection{考察}

\section{実験8}
\subsection{実験の目的、概要}
本実験では、プロセッサからステッピングモータを扱う。
最終的に、キーボードからモータを制御するプログラムを作成、実行する。

これによって、このプロセッサで提供されている機能でどのような操作ができ、できないのか確認することを目的とする。

\subsection{実験方法}

\subsection{実験結果}

\subsection{考察}

\section{実験8}
\subsection{実験の目的、概要}
本実験では、プロセッサからステッピングモータを扱う。
最終的に、キーボードからモータを制御するプログラムを作成、実行する。

これによって、このプロセッサで提供されている機能でどのような操作ができ、できないのか確認することを目的とする。

\subsection{実験方法}

\subsection{実験結果}

\subsection{考察}

\section{実験8}
\subsection{実験の目的、概要}
本実験では、プロセッサからステッピングモータを扱う。
最終的に、キーボードからモータを制御するプログラムを作成、実行する。

これによって、このプロセッサで提供されている機能でどのような操作ができ、できないのか確認することを目的とする。

\subsection{実験方法}

\subsection{実験結果}

\subsection{考察}


\end{document}
