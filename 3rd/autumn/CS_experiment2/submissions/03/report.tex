\documentclass[a4j,15pt,draft]{jsarticle}

% 数式
\usepackage{amsmath,amsfonts}
\usepackage{bm}
% 画像
\usepackage[dvipdfmx]{graphicx}

\usepackage{here}
\usepackage{url}

\usepackage{listingsutf8,jlisting} %日本語のコメントアウトをする場合jlistingが必要
%ここからソースコードの表示に関する設定
\lstset{
  basicstyle={\ttfamily},
  identifierstyle={\small},
  commentstyle={\smallitshape},
  keywordstyle={\small\bfseries},
  ndkeywordstyle={\small},
  stringstyle={\small\ttfamily},
  frame={tb},
  breaklines=true,
  columns=[l]{fullflexible},
  numbers=left,
  xrightmargin=0zw,
  xleftmargin=3zw,
  numberstyle={\scriptsize},
  stepnumber=1,
  numbersep=1zw,
  lineskip=-0.5ex
}

\begin{document}

\title{コンピュータ科学実験レポート}
\author{坪井正太郎(101830245)}
\date{\today}
\maketitle

\section*{はじめに}
この実験では、一部の命令が実装されていないプロセッサに、適切な命令を実装して、条件付きループ命令を含む動作を行えるようにする。

また、各実験では,シュミレータや論理合成のソフトウェアを使うために,以下の設定を行う。
端末を終了した場合,再度sourceコマンドを実行する。
\begin{lstlisting}[caption={設定の読み込み},label={設定の読み込み}]
  $ ln -s /pub1/jikken/eda3/cadsetup.bash.altera ~/
  $ source ~/cadsetup.bash.altera
\end{lstlisting}

HDLのコンパイルにはQuartus Primeを、機能レベルシュミレーションにはModel Simを使用した。

バイナリファイルの内容は、hexdumpコマンドによる。
一番左のカラムは、hexdumpの行数である。

\section*{各実験}
\section{実験5-1}
\subsection{実験の目的、概要}
この実験では、実験4-2で作成したプロセッサで関数呼び出しを行い、それらの動作を予想し、確かめる。

これによって、現在足りていない機能を確認することを目的とする。

\subsection{実験方法}

\subsection{実験結果}

\subsection{考察}

\section{実験5-2}
\subsection{実験の目的、概要}
実験4-2で作成したプロセッサにjal命令が足りないことが、実験5-1で確認できた。
本実験では、プロセッサにjal命令を追加実装し、動作を確認する。
その際、実験5-1での予想と実際を比較する。

これによって、関数呼び出し時のプロセッサの動作、レジスタに保存されるデータなどを確認することを目的とする。

\subsection{実験方法}
\subsection{追加設計}
main\_ctrl.vに以下の変更を加え、jal命令のオペコードと、実行されるときの制御信号を定義した。
\begin{lstlisting}[caption={main\_ctrl.vの追加設計},label={mainctrl.vの追加設計5-2}]
オペコードの定義
`define    JAL  6'b000011  //  jump and link (J 形式)

jp_sel モジュールへの制御信号の記述
+=4の値ではなく、命令文中のアドレスを選択する
assign  jp = ((op_code == `J) || (op_code == `JAL)) ? 1'b1 : 1'b0;

レジスタのwrite_enable制御信号の追加
`JAL:    reg_write_enable_tmp = 1'b1;

レジスタに流すデータのセレクト信号の追加
REG[31]に次のPC値が保存されるようになる
`JAL:    link_tmp = 1'b1;
\end{lstlisting}

\subsubsection{論理合成、ダウンロード}
以下の操作で、論理合成し、FPGAにダウンロードした。
\begin{lstlisting}[caption={論理合成、ダウンロード},label={論理合成、ダウンロード5-2}]
$  cp rom8x1024.mif ./mips_de10-lite/
$  cd ./mips_de10-lite/
$  quartus_sh --flow compile MIPS_Default
$  quartus_pgm MIPS_Default.cdf
\end{lstlisting}

実験5-1で予想した点と、ディスプレイに表示されるはずの文字について確認した。

\subsection{実験結果}
予想した点について結果は、以下のようになった。
\begin{itemize}
  \item 最初にPC=0x0074を実行した直後のREG[31]の値
        \begin{itemize}
          \item REGWRITED=00400078,index=0x1f(=31),write\_enable=1が読み取れた
          \item 予想通り、次に実行するPCの値が、REG[31]
        \end{itemize}
  \item 最初にPC=0x0074を実行した直後のPCの値
        \begin{itemize}
          \item 予想通りPC=0x=0x00a0であり、ジャンプしている
        \end{itemize}
\end{itemize}

画面上には、"HELLO!!"という文字列が表示された。

\subsection{考察}
jal命令を実装したことで、\$raにPCの値が退避され、即値で指定した値にジャンプできるようになった。
そのため、予想した点について正しい動作を確認することができた。
これは、my\_print関数の呼び出しに成功しているということであると考えられる。
実際に、画面上に文字列が表示されることが分かった。

この実験で、関数呼び出し時のプロセッサの動作、レジスタに保存されるデータを確認する事ができた。

\section{実験6-1}
\subsection{実験の目的、概要}

\subsection{実験方法}

\subsection{実験結果}

\subsection{考察}

\section{実験6-2}
\subsection{実験の目的、概要}
実験5-2で作成したプロセッサにjr命令が足りないことが、実験6-1で確認できた。
本実験では、プロセッサにjr命令を追加実装し、動作を確認する。
その際、実験6-1での予想と実際を比較する。

これによって、関数からの復帰時のプロセッサの動作、レジスタから読み取れるデータなどを確認することを目的とする。

\subsection{実験方法}

\subsection{追加設計}
cpu.vに、以下の変更を加え、jpr\_selモジュールを追加した。
\begin{lstlisting}[caption={cpu.vの追加設計},label={cpu.vの追加設計6-2}]
jpr_selの入出力ワイヤを定義
wire  [31:0]     jpr_sel_d0;  // jpr 選択回路モジュール データ 1
wire  [31:0]     jpr_sel_d1;  // jpr 選択回路モジュール データ 2
wire  [31:0]     jpr_sel_s;  // jpr 選択回路モジュール セレクト信号
wire  [31:0]     jpr_sel_y;  // jpr 選択回路モジュール 出力

32bitマルチプレクサモジュールをjpr_selとして宣言する
入出力はさっき定義したワイヤを使う
mux32_32_32  jpr_sel(jpr_sel_d0, jpr_sel_d1, jpr_sel_s, jpr_sel_y);

セレクタの出力をPCに接続する
assign pc_next = jpr_sel_y;
代わりに,割り当てられていたjp_sel_yをコメントアウトする
//assign pc_next = jp_sel_y;

セレクタの入力を割り当てる
assign jpr_sel_d0 = jp_sel_y;
assign jpr_sel_d1 = alu_ram_sel_y;
assign jpr_sel_s = jpr;
\end{lstlisting}

main\_ctrl.vに以下の変更を加え、jr命令が実行されるときの制御信号を定義した。
\begin{lstlisting}[caption={main\_ctrl.vの追加設計},label={mainctrl.vの追加設計6-2}]
jpr信号を出力する条件として、jrのファンクションコードを追加する。
|| Rfunc == 6'b001000)) ? 1'b1 : 1'b0;

レジスタの書き込み制御信号の出力条件を追加
if (Rfunc == 6'b001000) begin
  reg_write_enable_tmp = 1'b0;
end else begin
  reg_write_enable_tmp = 1'b1;
end
\end{lstlisting}

\subsection{実験結果}

\subsection{考察}

\section{実験7}
\subsection{実験の目的、概要}
本実験では、実験6-2で作成したプロセッサ上で、キーボード入力に応答を行うプログラムを動作させる。
このプログラムは、プロセッサに実装されていない命令を使用するため、正常に動作しない。

本実験では、このプログラムを、正常に動作させることを目的とする。

\subsection{実験方法1}
以下のようなプログラムを配置した。
\lstinputlisting[caption=sosuu.c,label=sosuu.c7]{src/07/sosuu.c}

\subsubsection{クロスコンパイル、メモリイメージファイルの作成}
以下の操作でクロスコンパイルし、メモリイメージファイルを作成した。
\begin{lstlisting}[caption={クロスコンパイル、メモリイメージファイルの作成},label={クロスコンパイル、メモリイメージファイルの作成5-1}]
$  cross_compile.sh sosuu.c
$  bin2v sosuu.bin
\end{lstlisting}

\subsubsection{命令列の確認、動作予想}
生成された、rom8x1024.mifを確認して、以下の点について結果を予測した。
\begin{itemize}
  \item 命令メモリの0x082の命令はどのような命令か
  \item 命令メモリの0x082の命令はsosuu\_check()のどの記述に対応するか
\end{itemize}

\subsubsection{論理合成、ダウンロード}
以下の操作で、論理合成し、FPGAにダウンロードした。
\begin{lstlisting}[caption={論理合成、ダウンロード},label={論理合成、ダウンロード6-1}]
$  cp rom8x1024.mif ./mips_de10-lite/
$  cd ./mips_de10-lite/
$  quartus_sh --flow compile MIPS_Default
$  quartus_pgm MIPS_Default.cdf
\end{lstlisting}

FPGA上で、HELLO, NUM=と表示されたら、"20"を入力し、その結果を確認した。

このプログラムは正しく動作しない。
この問題について、解決する方法を2つ考えた。

\subsection{実験結果1}
\subsubsection{命令列からの予想}
メモリイメージファイルから、以下のような予想をたてた。
\begin{itemize}
  \item 命令メモリの0x082の命令はどのような命令か
        \begin{itemize}
          \item 2つのレジスタの内容を符号なし整数と解釈して除算し、商はLO、余りはHIに格納する、divu命令
          \item rsに3,rtに2を指定しているため、LO=REG[3]/REG[2]、LO=REG[3]\%REG[2]
        \end{itemize}
  \item 命令メモリの0x082の命令はsosuu\_check()のどの記述に対応するか
        \begin{itemize}
          \item ソースコード\ref{sosuu.c7}の88行目 if ((kouho \% tester) == 0)に対応する
          \item RAMに0(false)を代入しており、変数をlwして++2していることから分かる
        \end{itemize}
\end{itemize}

\subsubsection{プログラムの実行}
以下のような表示となり、素数が表示されず、正しく実行されなかった。
\begin{lstlisting}[caption={sosuu.c実行結果1},label={sosuu.c実行結果17}]
HELLO
NUM=20
ECHO 20
03 05 08
NUM=
\end{lstlisting}

この問題について、以下のような解決方法を考えた。
\begin{itemize}
  \item プロセッサが実行できるようにプログラムを変更
  \item プロセッサに命令を足りない命令を追加
\end{itemize}

\subsection{実験方法2}
考案した解決方法のうち、今回は、「プロセッサが実行できるようにプログラムを変更」という方法で解決した。

\subsubsection{Cプログラムの変更}
sosuu.cのプログラムを以下のように変更した。
\lstinputlisting[caption=sosuu.c,label=sosuu.c27]{src/07/sosuu’.c}

\subsubsection{論理合成、ダウンロード}
以下の操作で、論理合成し、FPGAにダウンロードした。
\begin{lstlisting}[caption={論理合成、ダウンロード},label={論理合成、ダウンロード6-1}]
$  cp rom8x1024.mif ./mips_de10-lite/
$  cd ./mips_de10-lite/
$  quartus_sh --flow compile MIPS_Default
$  quartus_pgm MIPS_Default.cdf
\end{lstlisting}

FPGA上で実行し、プログラムの動作を確認した。

\subsubsection{実験結果2}
画面上には、以下のような結果が表示され、正しく素数を判定できていた。
\begin{lstlisting}[caption={実行結果2},label={実行結果27}]
HELLO
NUM=20
ECHO 20
03 05 07 11 13 17 19
NUM=
\end{lstlisting}

\subsection{考察}
このプロセッサには、divu命令が実装されていない。
そのため、正しくプログラムを実行することができなかったが、プログラムに変更することで、正しく動作させることに成功した。

必要な変更は、剰余を用いずにkouhoをtesterで割り切ることができるか判定することであった。
ここでは、testerにtesterを加えていき、kouhoを超える前に、kouho==testerとなるかどうかを判定した。
これによって、kouhoがtesterで割り切ることができるか判定できると考えた。
実際に、判定プログラムは正しく動作した。

\section{実験8}
\subsection{実験の目的、概要}
本実験では、プロセッサからステッピングモータを扱う。
最終的に、キーボードからモータを制御するプログラムを作成、実行する。

これによって、このプロセッサで提供されている機能を使用することで実現できるプログラムを作成することを目的とする。

\subsection{実験方法}
以下のようなプログラムを配置した。
\lstinputlisting[caption=motor.c,label=motor.c]{src/08/motor.c}

\subsubsection{クロスコンパイル、メモリイメージファイルの作成}
以下の操作でクロスコンパイルし、メモリイメージファイルを作成した。
\begin{lstlisting}[caption={クロスコンパイル、メモリイメージファイルの作成},label={クロスコンパイル、メモリイメージファイルの作成8}]
$  cross_compile.sh motor.c
$  bin2v motor.bin
\end{lstlisting}

\subsubsection{モーターの接続}
このプログラムでは、GPIOピンを経由して、ステッピングモータを使用するため、プログラムのダウンロード前に、制御回路とGPIOピンを、モータと制御回路を接続した。

\subsubsection{論理合成、ダウンロード}
以下の操作で、論理合成し、FPGAにダウンロードした。
\begin{lstlisting}[caption={論理合成、ダウンロード},label={論理合成、ダウンロード8}]
$  cp rom8x1024.mif ./mips_de10-lite/
$  cd ./mips_de10-lite/
$  quartus_sh --flow compile MIPS_Default
$  quartus_pgm MIPS_Default.cdf
\end{lstlisting}

FPGA上でプログラムを実行し、結果を確認した。

\subsubsection{ステッピングモータ制御プログラムの作成}
コンパイラと、プロセッサの対応する機能に注意して、motor.cを変更した。
\lstinputlisting[caption=motor.c',label=motor.c']{src/08/my_motor.c}

\subsubsection{クロスコンパイル、メモリイメージファイルの作成}
以下の操作でクロスコンパイルし、メモリイメージファイルを作成した。
\begin{lstlisting}[caption={クロスコンパイル、メモリイメージファイルの作成},label={クロスコンパイル、メモリイメージファイルの作成28}]
$  cross_compile.sh motor.c
$  bin2v motor.bin
\end{lstlisting}

\subsubsection{論理合成、ダウンロード}
以下の操作で、論理合成し、FPGAにダウンロードした。
\begin{lstlisting}[caption={論理合成、ダウンロード},label={論理合成、ダウンロード28}]
$  cp rom8x1024.mif ./mips_de10-lite/
$  cd ./mips_de10-lite/
$  quartus_sh --flow compile MIPS_Default
$  quartus_pgm MIPS_Default.cdf
\end{lstlisting}

ダウンロード後、FPGAで実行して動作を確認した。

\subsection{実験結果}
最初のmotor.cを実行した結果、モータは1方向に回転し続けた。

motor.cを変更し、実行した結果、以下のような結果になった。
(//でコメントアウトした部分で、モータの様子を記述した)
\begin{lstlisting}[caption={変更したmotor.cの実行結果},label={変更したmotor.cの実行結果}]
  STEP=200
  DIR?=1
  //最初の実行と同じ方向に、200ステップ分回転した
  DONE!
  STEP=100
  DIR?=0
  //最初の実行と逆の方向に、200ステップ分回転した
  DONE!
\end{lstlisting}

\subsection{考察}
変更前のmotor.cから、GPIOピンに、4bitのフラグを入力し、コイルの相を指定して励磁させていると考えられる。
プログラムでは、「$1000\rightarrow 0100\rightarrow 0010\rightarrow 0001$」の順で繰り返し励磁されていた。

変更後のプログラムでは、順方向の他に、逆方向への回転も実現するために、テーブルを定義した。
1248と、8421の2つのテーブルを用意し、4bitシフトして下位4bitをとることで励磁するビット数を決定した。
また、回転するステップ数を指定するため、ループ数を入力から指定することにした。

この実験でのプログラムの作成を通して、プロセッサに搭載されている機能の中でプログラムを作成することができた。


\end{document}
