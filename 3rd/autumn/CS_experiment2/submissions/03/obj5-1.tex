\section{実験5-1}
\subsection{実験の目的、概要}
この実験では、実験4-2で作成したプロセッサで画面上に文字を表示するCプログラムを実行する
その中で関数呼び出しを行い、動作を予想し、結果を確認する。

これによって、現在足りていない機能を確認することを目的とする。

\subsection{実験方法}
以下のプログラムを配置した。
% \lstinputlisting[caption=my\_print.c,label=myprint.c5-1]{}

\subsubsection{クロスコンパイル、メモリイメージファイルの作成}
以下の操作でクロスコンパイルし、メモリイメージファイルを作成した。
\begin{lstlisting}[caption={クロスコンパイル、メモリイメージファイルの作成},label={クロスコンパイル、メモリイメージファイルの作成5-1}]
$  cross_compile.sh my_print.c
$  bin2v my_print.bin
\end{lstlisting}

\subsubsection{命令列の確認、動作予想}
生成された、rom8x1024.mifを確認して、以下の点について結果を予測した。
\begin{itemize}
  \item 最初にPC=0x0074を実行した直後のREG[31]の値
  \item 最初にPC=0x0074を実行した直後のPCの値
\end{itemize}

\subsubsection{論理合成、ダウンロード}
以下の操作で、論理合成し、FPGAにダウンロードした。
\begin{lstlisting}[caption={論理合成、ダウンロード},label={論理合成、ダウンロード5-1}]
$  cp rom8x1024.mif ./mips_de10-lite/
$  cd ./mips_de10-lite/
$  quartus_sh --flow compile MIPS_Default
$  quartus_pgm MIPS_Default.cdf
\end{lstlisting}

クロックを手動モードで送り、70個ほどの命令を実行、予想した点と、ディスプレイに表示されるはずの文字について確認した。

\subsection{実験結果}
\subsubsection{命令列の確認、動作予想}
以下のようなメモリイメージファイルが生成された。
\lstinputlisting[caption=rom1024.mif,label=rom1024.mif]{}

命令列を確認して、このような予想をたてた。
\begin{itemize}
  \item 最初にPC=0x0074を実行した直後のREG[31]の値
  \begin{itemize}
    \item REG[31]=0x0078
  \end{itemize}
  \item 最初にPC=0x0074を実行した直後のPCの値
  \begin{itemize}
    \item PC=0x00a0
  \end{itemize}
\end{itemize}

\subsubsection{FPGAでの実行結果}
予想した点について結果は、以下のようになった。
\begin{itemize}
  \item 最初にPC=0x0074を実行した直後のREG[31]の値
  \begin{itemize}
    \item REGWRITED=00000000,WEN=0であり、レジスタへの書き込みは発生していない
  \end{itemize}
  \item 最初にPC=0x0074を実行した直後のPCの値
  \begin{itemize}
    \item PC=0x0078であり、ジャンプなどはしていない
  \end{itemize}
\end{itemize}

ディスプレイに文字は表示されなかった。

\subsection{考察}
