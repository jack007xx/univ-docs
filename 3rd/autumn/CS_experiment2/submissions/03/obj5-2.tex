\section{実験5-2}
\subsection{実験の目的、概要}
実験4-2で作成したプロセッサにjal命令が足りないことが、実験5-1で確認できた。
本実験では、プロセッサにjal命令を追加実装し、動作を確認する。
その際、実験5-1での予想と実際を比較する。

これによって、関数呼び出し時のプロセッサの動作、レジスタに保存されるデータなどを確認することを目的とする。

\subsection{実験方法}
\subsection{追加設計}
main\_ctrl.vに以下の変更を加え、jal命令のオペコードと、実行されるときの制御信号を定義した。
\begin{lstlisting}[caption={main\_ctrl.vの追加設計},label={mainctrl.vの追加設計5-2}]
オペコードの定義
`define    JAL  6'b000011  //  jump and link (J 形式)

jp_sel モジュールへの制御信号の記述
+=4の値ではなく、命令文中のアドレスを選択する
assign  jp = ((op_code == `J) || (op_code == `JAL)) ? 1'b1 : 1'b0;

レジスタのwrite_enable制御信号の追加
`JAL:    reg_write_enable_tmp = 1'b1;

レジスタに流すデータのセレクト信号の追加
REG[31]に次のPC値が保存されるようになる
`JAL:    link_tmp = 1'b1;
\end{lstlisting}

\subsubsection{論理合成、ダウンロード}
以下の操作で、論理合成し、FPGAにダウンロードした。
\begin{lstlisting}[caption={論理合成、ダウンロード},label={論理合成、ダウンロード5-2}]
$  quartus_sh --flow compile MIPS_Default
$  quartus_pgm MIPS_Default.cdf
\end{lstlisting}

実験5-1で予想した点と、ディスプレイに表示されるはずの文字について確認した。

\subsection{実験結果}
予想した点について結果は、以下のようになった。
\begin{itemize}
  \item 最初にPC=0x0074を実行した直後のREG[31]の値
        \begin{itemize}
          \item REGWRITED=00400078,index=0x1f(=31),write\_enable=1が読み取れた
          \item 予想通り、次に実行するPCの値が、REG[31]
        \end{itemize}
  \item 最初にPC=0x0074を実行した直後のPCの値
        \begin{itemize}
          \item 予想通りPC=0x=0x00a0であり、ジャンプしている
        \end{itemize}
\end{itemize}

画面上には、"HELLO!!"という文字列が表示された。

\subsection{考察}
jal命令を実装したことで、\$raにPCの値が退避され、即値で指定した値にジャンプできるようになった。
そのため、予想した点について正しい動作を確認することができた。
これは、my\_print関数の呼び出しに成功しているということであると考えられる。
実際に、画面上に文字列が表示されることが分かった。

この実験で、関数呼び出し時のプロセッサの動作、レジスタに保存されるデータを確認する事ができた。
