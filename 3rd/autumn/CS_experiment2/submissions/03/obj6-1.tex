\section{実験6-1}
\subsection{実験の目的、概要}
実験5-2でjal命令を追加したプロセッサ上で、キーボードからの入力を受け取るCプログラムを実行する。
その中で関数からの復帰を行い、それらの動作を予想し、結果を確認する。

これによって、現在足りていない機能を確認することを目的とする。

\subsection{実験方法}
以下のプログラムを配置した。
\lstinputlisting[caption=my\_scan.c,label=myscan.c6-1]{src/06/my_scan.c}

\subsubsection{クロスコンパイル、メモリイメージファイルの作成}
以下の操作でクロスコンパイルし、メモリイメージファイルを作成した。
\begin{lstlisting}[caption={クロスコンパイル、メモリイメージファイルの作成},label={クロスコンパイル、メモリイメージファイルの作成5-1}]
$  cross_compile.sh my_scan.c
$  bin2v my_scan.bin
\end{lstlisting}

\subsubsection{命令列の確認、動作予想}
生成された、rom8x1024.mifを確認して、以下の点について結果を予測した。
\begin{itemize}
  \item 最初にPC=0x007cを実行したときの,REG[31]の値
  \item 最初にPC=0x0804を実行した直後の,PCの値
\end{itemize}

\subsubsection{論理合成、ダウンロード}
以下の操作で、論理合成し、FPGAにダウンロードした。
\begin{lstlisting}[caption={論理合成、ダウンロード},label={論理合成、ダウンロード6-1}]
$  cp rom8x1024.mif ./mips_de10-lite/
$  cd ./mips_de10-lite/
$  quartus_sh --flow compile MIPS_Default
$  quartus_pgm MIPS_Default.cdf
\end{lstlisting}

クロックを手動モードで送り、70個ほどの命令を実行、予想した点と、ディスプレイに表示されるはずの文字について確認した。

\subsection{実験結果}
\subsubsection{命令列の確認、動作予想}
メモリイメージファイルが生成された。
% \lstinputlisting[caption=rom1024.mif,label=rom1024.mif]{src/06/rom8x1024.mif}

生成された命令列を確認して、このような予想をたてた。
\begin{itemize}
  \item 最初にPC=0x007cを実行したときの,REG[31]の値
        \begin{itemize}
          \item 
        \end{itemize}
  \item 最初にPC=0x0804を実行した直後の,PCの値
\end{itemize}

\subsubsection{FPGAでの実行結果}
予想した点について結果は、以下のようになった。
\begin{itemize}
  \item 最初にPC=0x0074を実行した直後のREG[31]の値
        \begin{itemize}
          \item REGWRITED=0x00400080,IDX=0x1f,WEN=1となり、\$raにPCの値が退避された
        \end{itemize}
  \item 最初にPC=0x0804を実行した直後の,PCの値
        \begin{itemize}
          \item PC=0x00400808
        \end{itemize}
\end{itemize}

画面上には、"HELLO!!"のみ表示された。

\subsection{考察}
予想した点について、PC=0x0074の命令は正しく動作しているが、PC=0x0804の命令については正しく実行できていないということが分かる。
これは、プロセッサにjr命令が実装されていないためであると考えられる。

このプロセッサには、jal命令は実装されているが、jr命令が実装されていないので、関数に入ることはできても、関数から戻ることができない。
そのため、プログラム中のmy\_print関数の実行はできるが、戻ることができないため、次の手続きに進むことができないと考えられる。
