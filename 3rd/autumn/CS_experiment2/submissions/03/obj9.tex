\section{実験9}
\subsection{実験の目的、概要}
実験8で使用したプロセッサには、乗算命令multが実装されておらず、乗算を使用したプログラムは動作しない。
この実験では、乗算結果を決められたhi,loレジスタに保存するmult命令と、loレジスタから乗算結果の下位半分のbitを取得するmflo命令を実装する。

これによって、hi,loレジスタを使用した演算の動作を確認することを目的とする。

\subsection{実験方法}
以下のプログラムを配置した。
\lstinputlisting[caption=mult.c,label=mult.c]{src/09/mlut.c}

\subsubsection{bin2vの拡張}
bin2v.tar.gzを配置して、
tar xvfz ./bin2v.tar.gz で、bin2vのソースコードとMakefileを取得した。

bin2vに以下のような変更を行った。
\begin{lstlisting}[caption={bin2vの拡張},label={bin2vの拡張}]
// コメント追加、functコード追加、outputのためのコメント追加

// コメント追加
  ‘define R 6’b000000 R 形式 (add, addu, sub, subu, and, or, slt, jalr, jr, mult, mflo)

// コメント追加
  MULT(op = 000000, func = 011000)
  MULT {Hi, Lo} <= REG[rs] * REG[rt]; MULT rs,rt
  
// コメント追加
  MFLO(op = 000000, func = 010010)
	MFLO REG[rd] <= Lo; MFLO rd

// functコードの定義追加
  #define    MULT 24
  #define    MFLO 18

// functがmultのものだった場合の、rom8x1024_sim.v生成用の記述
  case MULT:
    fprintf(outfp, "10'h%03x: data = 32'h%08x; // %08x: MULT, ", rom_addr, wd, cmt_addr);
    fprintf(outfp, "{Hi, Lo}<=REG[%d]*REG[%d];\n", rs, rt);
    break;

// functがmfloのものだった場合の、rom8x1024_sim.v生成用の記述
    case MFLO:
    fprintf(outfp, "10'h%03x: data = 32'h%08x; // %08x: MFLO, ", rom_addr, wd, cmt_addr);
    fprintf(outfp, "REG[%d]<=Lo;\n", rd);
    break;

// functがmultのものだった場合の、rom8x1024_sim.v生成用の記述
    case MULT:
    fprintf(outfp, "%03x : %08x ; %% %08x: MULT, ", rom_addr, wd, cmt_addr);
    fprintf(outfp, "{Hi, Lo}<=REG[%d]*REG[%d]; %%\n",rs, rt);
    break;

// functがmfloのものだった場合の、rom8x1024_sim.v生成用の記述 
  case MFLO:
    fprintf(outfp, "%03x : %08x ; %% %08x: MFLO, ", rom_addr, wd, cmt_addr);
    fprintf(outfp, "REG[%d]<=Lo; %%\n",rd);
    break;
\end{lstlisting}

変更後、makeコマンドでmakeした。

\subsubsection{クロスコンパイル、メモリイメージファイルの作成}
以下の操作でクロスコンパイルし、メモリイメージファイルを作成した。
\begin{lstlisting}[caption={クロスコンパイル、メモリイメージファイルの作成},label={クロスコンパイル、メモリイメージファイルの作成9}]
$  cross_compile.sh mult.c
$  ./bin2v/bin2v mult.bin
\end{lstlisting}

\subsubsection{cpu.vの追加設計}
\begin{lstlisting}[caption={cpu.vの変更},label={cpu.vの変更9}]
  aluインスタンスの入出力を同期回路として定義し直した。
  alu  alua(alu_a, alu_b, alu_ctrl, alu_y, alu_comp);
  ↓
  alu alua(clock, reset, alu_a, alu_b, alu_ctrl, alu_y, alu_comp);
\end{lstlisting}

\subsubsection{main\_ctrl.vの追加設計}
\begin{lstlisting}[caption={main\_ctri.vの変更},label={mainctri.vの変更9}]
  // mult, mflo 命令に関するコメントの追加
  R 形式
  MULT(op = 000000, func = 011000)
  MULT {Hi, Lo} <= REG[rs] * REG[rt]; MULT rs,rt
  MFLO(op = 000000, func = 010010)
  MFLO REG[rd] <= Lo; MFLO rd
\end{lstlisting}

\subsubsection{alu.vの追加設計}
\begin{lstlisting}[caption={alu.vの変更},label={alu.vの変更9}]
  // コメントの追加
  // mult(multiply)
  // mflo(move from Lo)
  
  // alu制御コードのコメントの追加
  // 1011, mult
  // 1100, mflo
  
  // ALU制御コードの定義
‘define ALU_MULT 4’b1011
‘define ALU_MFLO 4’b1100
  
  // ALUモジュールの入力ポートの拡張
  // clock,reset信号の追加
  module alu (alu_a, alu_b, alu_ctrl, alu_y, alu_comp);
  // を
  module alu (clock, reset, alu_a, alu_b, alu_ctrl, alu_y, alu_comp);
  // に変更
  
  // mult命令実行時に alu_a * alu_b の結果を {hi, lo}に格納する記述の追加
  input clock, reset; // 入力 クロック, リセット
  reg [31:0] hi; //上位
  reg [31:0] lo; //下位
  always @(posedge clock or negedge reset) begin
    if (reset == 1’b0) begin
      hi <= 32’h00000000;
      lo <= 32’h00000000;
    end else begin
      {hi, lo} <= (alu_ctrl == ‘ALU_MULT) ? alu_a * alu_b : {hi, lo};
    end
  end
  
  // mflo命令実行時に {hi, lo} の lo を result に出力する記述の追加
‘ALU_MFLO: begin
    result <= lo;
  end
  
\end{lstlisting}

\subsubsection{alu\_ctrl.vの追加設計}
\begin{lstlisting}[caption={alu\_ctrl.v},label={aluctrl.v9}]
  // mult,mflo命令用のALU制御コードの定義
  `define     ALU_CTRL_MULT 4'b1011
  `define     ALU_CTRL_MFLO 4'b1100
  
  // mult,mflo命令用のALU制御コードについてのコメント追加
  // mult(multiply), 010, 011000, 1011
  // mflo(move from lo), 010, 010010, 1100
  
  // 実行する命令がmult,mflo命令のとき、mult,mflo命令用のALU制御コードを生成する処理の追加
  end else if (func == 6'b011000) begin // func=MULT
    y = `ALU_CTRL_MULT;
  end else if (func == 6'b010010) begin // func=MFLO
    y = `ALU_CTRL_MFLO;
\end{lstlisting}

\subsubsection{論理合成、ダウンロード}
以下の操作で、論理合成し、FPGAにダウンロードした。
\begin{lstlisting}[caption={論理合成、ダウンロード},label={論理合成、ダウンロード9}]
$  cp rom8x1024.mif ./mips_de10-lite/
$  cd ./mips_de10-lite/
$  quartus_sh --flow compile MIPS_Default
$  quartus_pgm MIPS_Default.cdf
\end{lstlisting}

FPGA上でプログラムを実行し、結果を確認した。

\subsection{実験結果}
\subsubsection{メモリイメーファイルの生成結果}
生成されたメモリイメーファイルのうち、追加実装した部分に関わるものは以下のようになった。
\begin{lstlisting}[caption={メモリイメーファイルの追加実装に関わる部分},label={メモリイメーファイルの追加実装に関わる部分}]
  050 : 00620018 ; % 00400140: MULT, {Hi, Lo}<=REG[3]*REG[2]; %
  051 : 00001012 ; % 00400144: MFLO, REG[2]<=Lo; %
\end{lstlisting}

\subsubsection{mult.cの実行結果}
mult.cを実行した結果、以下のようになった。
\begin{lstlisting}[caption={mult.cの実行結果},label={mult.cの実行結果}]
  HELLO
  NUM=4
  echo 4
  16
  NUM=6
  echo 6
  36
\end{lstlisting}

\subsection{考察}
\subsubsection{bin2vの追加実装}
bin2vの追加実装ではコメント追加のほか、命令コードの機能コードがmult,mflo命令のものだった場合のイメージファイルの生成コードを追加した。
結果に示したように、適切なコメントを挿入していた。

\subsubsection{プロセッサの追加実装}
cpu.vの追加実装では、クロックとリセット信号をALUに追加した。
これは、演算結果を保持するHi,Loレジスタの更新、消去のためであると考えられる。

alu.vの追加実装では、コントローラから入力される制御コードとしてmult,mflo命令のものを追加した。
alu内部では、クロック信号の立ち上がりで、制御コードがmultだった場合、乗算の結果をhiとloの連接に入力し、その他のときは両方のレジスタを同じ値でリライトしている。
また、リセット信号の立ち下がりでは、Hi,Lo両方を0でリセットしている。

mflo命令の制御コードを受けた場合、結果にはLoレジスタの値を入力している。

main\_ctrl.v、alu\_ctrl.vでは、命令に対応した制御コードを出力している。

\subsubsection{mult.cの実行結果}
mult.cを実行した結果から、プログラムは乗算命令を正しく実行できていることが分かる。
