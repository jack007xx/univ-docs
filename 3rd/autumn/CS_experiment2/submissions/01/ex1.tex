
\subsection{実験の目的,概要}
1桁のBCDコードを出力するBCDカウンタを設計し,それをもとに2桁のBCDコード出力するBCDカウンタを階層設計によって設計する。
設計した2つの回路で,それぞれ機能レベルシュミレーションと論理合成を実行し,入出力の正しさと合成による結果を確認する。

ICE計算機上の,ModelSim,Quartusを使用する。

この実験で,簡単な順序回路の設計方法と,それをもとにした階層設計の方法を実践し,習得することを目的にする。

\subsection{実験方法}
\subsection*{課題1-1}
\subsubsection{回路のHDL記述}
以下のような回路記述をbcd1.v,テストベンチをtest\_bcd1.vとして作成する。
\lstinputlisting[caption=bcd1.v,label=bcd1.v]{./src/bcd/bcd1.v}
\lstinputlisting[caption=test\_bcd1.v,label=testbcd1.v]{./src/bcd/test_bcd1.v}

\subsubsection{機能レベルシュミレーション}
作成したテストベンチをもとに,ModelSimで信号波形を出力する。
入出力の値が仕様通りの真理値表と一致することを確認する。

\subsubsection{論理合成}
以下の2ファイルを作成し,配置する。
\lstinputlisting[caption=bcd1.qpf,label=bcd1.qpf]{./src/bcd/bcd1.qpf}
\lstinputlisting[caption=bcd1.qsf,label=bcd1.qsf]{./src/bcd/bcd1.qsf}

作成した回路記述をQuartusでコンパイルし,論理合成とレイアウトを行う。回路構成やロジックエレメント数,遅延時間について確認する。

\subsection*{課題1-2}
以下のような回路記述をbcd2.v,テストベンチをtest\_bcd2.vとして作成する。
\lstinputlisting[caption=bcd2.v,label=bcd2.v]{./src/bcd/bcd2.v}
\lstinputlisting[caption=test\_bcd2.v,label=testbcd2.v]{./src/bcd/test_bcd2.v}

\subsubsection{機能レベルシュミレーション}
作成したテストベンチをもとに,ModelSimで信号波形を出力する。
入出力の値が仕様通りの真理値表と一致することを確認する。

\subsubsection{論理合成}
以下の2ファイルを作成し,配置する。
\lstinputlisting[caption=bcd2.qpf,label=bcd2.qpf]{./src/bcd/bcd2.qpf}
\lstinputlisting[caption=bcd2.qsf,label=bcd2.qsf]{./src/bcd/bcd2.qsf}

作成した回路記述をQuartusでコンパイルし,論理合成とレイアウトを行う。回路構成やロジックエレメント数,遅延時間について確認する。

\subsection{実験結果}
\subsubsection{回路のHDL記述}
\subsubsection{機能レベルシュミレーション}
\subsubsection{論理合成}
\subsection{考察}
