
\section{実験1}
\subsection{実験の目的,概要}
本実験では,2入力1出力のセレクタ回路を設計し,シュミレーションや論理合成を行って,FPGAボードでの動作を確認する。
これによってHDLによる記述,シュミレータの使い方,回路をFPGA上で動かすための方法を確認する。

入力はD0,D1(1bitデータ),S1(1bitセレクト信号)。
出力はY(1bit)。
セレクタ信号が0ならばD0のデータを,1ならばD1のデータを出力する。

ICE計算機上の,ModelSim,Quartus,FPGAボードはDE10-Liteを使用する。

\subsection{実験方法}
\subsubsection{HDLでの回路記述}
以下のような回路記述をmux.v,テストベンチをtest\_mux.vとして作成する。
\lstinputlisting[caption=mux.v,label=mux.v]{./src/mux21/mux21.v}
\lstinputlisting[caption=test\_mux.v,label=testmux.v]{./src/mux21/test_mux21.v}

\subsubsection{機能レベルシュミレーション}
作成したテストベンチをもとに,ModelSimで信号波形を出力する。
入出力の値が仕様通りの真理値表と一致することを確認する。

\subsubsection{コンパイル}
以下の2ファイルを作成し,配置する。
\lstinputlisting[caption=mux21.qpf,label=mux21.qpf]{./src/mux21/mux21.qpf}
\lstinputlisting[caption=mux21.qsf,label=mux21.qsf]{./src/mux21/mux21.qsf}

作成した回路記述をQuartusでコンパイルし,論理合成とレイアウトを行う。回路構成やロジックエレメント数,遅延時間について確認する。

\subsubsection{FPGAボードでの回路実現}
計算機にFPGAボードを接続し,dmesgコマンドで接続を確認する。
以下のような設定ファイルを配置し,`quartus-pgm mux21.cdf'を実行して,接続したFPGAにダウンロードする。
\lstinputlisting[caption=mux21.cdf ダウンロード設定ファイル,label=mux21.cdf ダウンロード設定ファイル]{./src/mux21/mux21.cdf}

FPGAが仕様通りに動作するかを確認する。

\subsection{実験結果}
\subsubsection{回路のHDL記述}
\subsubsection{機能レベルシュミレーション}
\subsubsection{論理合成}
\subsubsection{FPGAボードでの動作結果}
\subsection{考察}
