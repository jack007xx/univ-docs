
\subsection{実験の目的,概要}
本実験では,16ビット加算回路を順序回路で記述し,機能レベルシュミレーションと論理合成を実行する。
これによって,出力値が仕様を満たしているかどうか,論理合成による回路構成を確認する。

入力は,clk(1bitクロック),reset(1bitリセット信号),x,y(16bitオペランド),y(1bit桁上げ入力)。
出力は,sum(16bit演算結果),cout(1bit桁上げ出力)。
ただし,入力値の演算結果は次のクロックの立ち上がりで出力に反映させる。

ICE計算機上の,ModelSim,Quartusを使用する。

順序回路の設計法と回路合成結果の確認を行い,実験2の結果と比較することで,等価な回路が設計によって,どのように異なるかを考察することを目的とする。

\subsection{実験方法}
\subsubsection{回路のHDL記述}
以下のような回路記述をadder16s.v,テストベンチをtest\_adder16s.vとして作成する。
\lstinputlisting[caption=adder16s.v,label=adder16s.v]{./src/adder16s/adder16s.v}
\lstinputlisting[caption=test\_adder16s.v,label=testadder16s.v]{./src/adder16s/test_adder16s.v}

\subsubsection{機能レベルシュミレーション}
作成したテストベンチをもとに,ModelSimで信号波形を出力する。
入出力の値が仕様通りの真理値表と一致することを確認する。

\subsubsection{論理合成}
以下の2ファイルを作成し,配置する。
\lstinputlisting[caption=adder16s.qpf,label=adder16s.qpf]{./src/adder16s/adder16s.qpf}
\lstinputlisting[caption=adder16s.qsf,label=adder16s.qsf]{./src/adder16s/adder16s.qsf}

作成した回路記述をQuartusでコンパイルし,論理合成とレイアウトを行う。回路構成やロジックエレメント数,遅延時間について確認する。
 
\subsection{実験結果}
\subsubsection{回路のHDL記述}
\subsubsection{機能レベルシュミレーション}
\subsubsection{論理合成}
\subsection{考察}
