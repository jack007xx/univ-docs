
\subsection{実験の目的,概要}
本実験では,0011および0010が入力されるたびに,1を出力する系列検出回路を設計する。
このとき,状態器械による順序回路記述で設計する。
作成した回路は,機能レベルシュミレーションと論理合成を行い,結果を確認する。

ICE計算機上の,ModelSim,Quartusを使用する。

この実験で,オートマトンの設計を回路にどう反映させるかの方法について,習得することを目的にする。

\subsection{実験方法}
\subsubsection{回路のHDL記述}
以下のような回路記述をm.v,テストベンチをtest\_m.vとして作成する。
\lstinputlisting[caption=m.v,label=m.v]{./src/m/m.v}
\lstinputlisting[caption=test\_m.v,label=testm.v]{./src/m/test_m.v}

\subsubsection{機能レベルシュミレーション}
作成したテストベンチをもとに,ModelSimで信号波形を出力する。
入出力の値が仕様通りの真理値表と一致することを確認する。

\subsubsection{論理合成}
以下の2ファイルを作成し,配置する。
\lstinputlisting[caption=m.qpf,label=m.qpf]{./src/m/m.qpf}
\lstinputlisting[caption=m.qsf,label=m.qsf]{./src/m/m.qsf}

作成した回路記述をQuartusでコンパイルし,論理合成とレイアウトを行う。回路構成やロジックエレメント数,遅延時間について確認する。

\subsection{実験結果}
\subsubsection{回路のHDL記述}
\subsubsection{機能レベルシュミレーション}
\subsubsection{論理合成}
\subsection{考察}
