
\subsection{実験の目的,概要}
本実験では,16ビット加算回路を組み合わせ回路で記述し,機能レベルシュミレーションと論理合成を実行する。
これによって,出力値が仕様を満たしているかどうか,論理合成による回路構成を確認する。

入力は,x,y(16bitオペランド),y(1bit桁上げ入力)。
出力は,sum(16bit演算結果),cout(1bit桁上げ出力)。

ICE計算機上の,ModelSim,Quartusを使用する。

組み合わせ回路の設計法と回路合成結果の確認を目的とする。

\subsection{実験方法}
\subsubsection{回路のHDL記述}
以下のような回路記述をadder16.v,テストベンチをtest\_adder16.vとして作成する。
\lstinputlisting[caption=adder16.v,label=adder16.v]{./src/adder16/adder16.v}
\lstinputlisting[caption=test\_adder16.v,label=testadder16.v]{./src/adder16/test_adder16.v}

\subsubsection{機能レベルシュミレーション}
作成したテストベンチをもとに,ModelSimで信号波形を出力する。
入出力の値が仕様通りの真理値表と一致することを確認する。

\subsubsection{論理合成}
以下の2ファイルを作成し,配置する。
\lstinputlisting[caption=adder16.qpf,label=adder16.qpf]{./src/adder16/adder16.qpf}
\lstinputlisting[caption=adder16.qsf,label=adder16.qsf]{./src/adder16/adder16.qsf}

作成した回路記述をQuartusでコンパイルし,論理合成とレイアウトを行う。回路構成やロジックエレメント数,遅延時間について確認する。
 
\subsection{実験結果}
\subsubsection{回路のHDL記述}
\subsubsection{機能レベルシュミレーション}
\subsubsection{論理合成}
\subsection{考察}
