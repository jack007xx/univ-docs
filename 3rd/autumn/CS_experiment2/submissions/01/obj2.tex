
\subsection{実験の目的,概要}
本実験では,16ビット加算回路を組み合わせ回路で記述し,機能レベルシュミレーションと論理合成を実行する。
これによって,出力値が仕様を満たしているかどうか,論理合成による回路構成を確認する。

入力は,x,y(16bitオペランド),y(1bit桁上げ入力)。
出力は,sum(16bit演算結果),cout(1bit桁上げ出力)。

ICE計算機上の,ModelSim,Quartusを使用する。

組み合わせ回路の設計法と回路合成結果の確認を目的とする。

\subsection{実験方法}
\subsubsection{回路のHDL記述}
以下のような回路記述をadder16.v,テストベンチをtest\_adder16.vとして作成する。
\lstinputlisting[caption=adder16.v,label=adder16.v]{./src/adder16/adder16.v}
\lstinputlisting[caption=test\_adder16.v,label=testadder16.v]{./src/adder16/test_adder16.v}

\subsubsection{機能レベルシュミレーション}
作成したテストベンチをもとに,ModelSimで信号波形を出力する。
入出力の値が仕様通りの真理値表と一致することを確認する。

\subsubsection{論理合成}
以下の2ファイルを作成し,配置する。
\lstinputlisting[caption=adder16.qpf,label=adder16.qpf]{./src/adder16/adder16.qpf}
\lstinputlisting[caption=adder16.qsf,label=adder16.qsf]{./src/adder16/adder16.qsf}

作成した回路記述をQuartusでコンパイルし,論理合成とレイアウトを行う。回路構成やロジックエレメント数,遅延時間について確認する。
 
\subsection{実験結果}
\subsubsection{機能レベルシュミレーション}
ModelSimでの入出力波形は以下のようになった。
(キャプチャーでは値が見えないので,赤数字で10進の値を追記した。)

\begin{figure}[H]
  \centering
  \includegraphics[width=\linewidth]{./src/adder16/adder16_wave21.png}
  \caption{adder16の波形}
\end{figure}

足し算の結果は正しく出力されていることが確認できた。

\subsubsection{論理合成}
論理合成の結果,以下のような回路が作られた。

\begin{figure}[H]
  \centering
  \includegraphics[width=\linewidth]{./src/adder16/adder16_print.png}
  \caption{adder16の回路}
\end{figure}

ロジックエレメント数は19だった。

回路全体の遅延時間は,以下のようになった。

\begin{figure}[H]
  \centering
  \includegraphics[width=\linewidth]{./src/adder16/adder16Timing.png}
  \caption{adder16の遅延時間}
\end{figure}

\subsection{考察}
\subsubsection{回路のHDL記述}
コード\ref{adder16.v}では,16bitの入出力を定義して,assign句で出力に演算結果を割り当てている。
17bit連接に対して足し算の結果を割り当てているため,結果が16bitに収まらない場合,coutの値は1になる。

コード\ref{testadder16.v}ではテストベンチとして,入力をすべてゼロに初期化し,10nsごとにxを100づつ加算し,5nsごとにyを300づつ加算している。

\subsubsection{機能レベルシュミレーション}
シュミレーションの結果は仕様を満たす動作だった。
キャプチャの範囲の外で,coutに桁上げが出力されていることも確認できた。

\subsubsection{論理合成}
今回の合成結果では,一桁ごとに桁上りを含めた3入力の全加算器で加算し,直列につなげていた。
また,遅延時間は入出力の前後に集中していた。
