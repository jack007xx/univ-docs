\documentclass[a4paper,15pt]{jsarticle}

% 数式
\usepackage{amsmath,amsfonts}
\usepackage{bm}
% 画像
\usepackage[dvipdfmx]{graphicx}
\usepackage{here}

\usepackage{listingsutf8,jlisting} %日本語のコメントアウトをする場合jlistingが必要
%ここからソースコードの表示に関する設定
\lstset{
  basicstyle={\ttfamily},
  identifierstyle={\small},
  commentstyle={\smallitshape},
  keywordstyle={\small\bfseries},
  ndkeywordstyle={\small},
  stringstyle={\small\ttfamily},
  frame={tb},
  breaklines=true,
  columns=[l]{fullflexible},
  numbers=left,
  xrightmargin=0zw,
  xleftmargin=3zw,
  numberstyle={\scriptsize},
  stepnumber=1,
  numbersep=1zw,
  lineskip=-0.5ex
}

\begin{document}

\title{コンピュータ科学実験レポート}
\author{坪井正太郎(101830245)}
\date{\today}
\maketitle
\section*{概要}

\section*{はじめに}
各実験で,シュミレータや論理合成のソフトウェアを使うために,以下の設定を行う。
端末を終了した場合,再度sourceコマンドを実行する。
\begin{lstlisting}[caption={設定の読み込み},label={設定の読み込み}]
  $ ln -s /pub1/jikken/eda3/cadsetup.bash.altera ~/
  $ source ~/cadsetup.bash.altera
\end{lstlisting}

\section*{各実験}

\section{実験1}
\subsection{実験の目的,概要}
本実験では,2入力1出力のセレクタ回路を設計し,シュミレーションや論理合成を行って,FPGAボードでの動作を確認する。
これによってHDLによる記述,シュミレータの使い方,回路をFPGA上で動かすための方法を確認する。

入力はD0,D1(1bitデータ),S1(1bitセレクト信号)。
出力はY(1bit)。
セレクタ信号が0ならばD0のデータを,1ならばD1のデータを出力する。

ICE計算機上の,ModelSim,Quartus,FPGAボードはDE10-Liteを使用する。

\subsection{実験方法}
\subsubsection{HDLでの回路記述}
以下のような回路記述をmux.v,テストベンチをtest\_mux.vとして作成する。
\lstinputlisting[caption=mux.v,label=mux.v]{./src/mux21/mux21.v}
\lstinputlisting[caption=test\_mux.v,label=testmux.v]{./src/mux21/test_mux21.v}

\subsubsection{機能レベルシュミレーション}
作成したテストベンチをもとに,ModelSimで信号波形を出力する。
入出力の値が仕様通りの真理値表と一致することを確認する。

\subsubsection{コンパイル}
以下の2ファイルを作成し,配置する。
\lstinputlisting[caption=mux21.qpf,label=mux21.qpf]{./src/mux21/mux21.qpf}
\lstinputlisting[caption=mux21.qsf,label=mux21.qsf]{./src/mux21/mux21.qsf}

作成した回路記述をQuartusでコンパイルし,論理合成とレイアウトを行う。回路構成やロジックエレメント数,遅延時間について確認する。

\subsubsection{FPGAボードでの回路実現}
計算機にFPGAボードを接続し,dmesgコマンドで接続を確認する。
以下のような設定ファイルを配置し,`quartus-pgm mux21.cdf'を実行して,接続したFPGAにダウンロードする。
\lstinputlisting[caption=mux21.cdf ダウンロード設定ファイル,label=mux21.cdf ダウンロード設定ファイル]{./src/mux21/mux21.cdf}

FPGAが仕様通りに動作するかを確認する。

\subsection{実験結果}
\subsubsection{回路のHDL記述}
\subsubsection{機能レベルシュミレーション}
\subsubsection{論理合成}
\subsubsection{FPGAボードでの動作結果}
\subsection{考察}


\section{実験2}

\subsection{実験の目的,概要}
本実験では,16ビット加算回路を組み合わせ回路で記述し,機能レベルシュミレーションと論理合成を実行する。
これによって,出力値が仕様を満たしているかどうか,論理合成による回路構成を確認する。

入力は,x,y(16bitオペランド),y(1bit桁上げ入力)。
出力は,sum(16bit演算結果),cout(1bit桁上げ出力)。

ICE計算機上の,ModelSim,Quartusを使用する。

組み合わせ回路の設計法と回路合成結果の確認を目的とする。

\subsection{実験方法}
\subsubsection{回路のHDL記述}
以下のような回路記述をadder16.v,テストベンチをtest\_adder16.vとして作成する。
\lstinputlisting[caption=adder16.v,label=adder16.v]{./src/adder16/adder16.v}
\lstinputlisting[caption=test\_adder16.v,label=testadder16.v]{./src/adder16/test_adder16.v}

\subsubsection{機能レベルシュミレーション}
作成したテストベンチをもとに,ModelSimで信号波形を出力する。
入出力の値が仕様通りの真理値表と一致することを確認する。

\subsubsection{論理合成}
以下の2ファイルを作成し,配置する。
\lstinputlisting[caption=adder16.qpf,label=adder16.qpf]{./src/adder16/adder16.qpf}
\lstinputlisting[caption=adder16.qsf,label=adder16.qsf]{./src/adder16/adder16.qsf}

作成した回路記述をQuartusでコンパイルし,論理合成とレイアウトを行う。回路構成やロジックエレメント数,遅延時間について確認する。
 
\subsection{実験結果}
\subsubsection{機能レベルシュミレーション}
ModelSimでの入出力波形は以下のようになった。
(キャプチャーでは値が見えないので,赤数字で10進の値を追記した。)

\begin{figure}[H]
  \centering
  \includegraphics[width=\linewidth]{./src/adder16/adder16_wave21.png}
  \caption{adder16の波形}
\end{figure}

足し算の結果は正しく出力されていることが確認できた。

\subsubsection{論理合成}
論理合成の結果,以下のような回路が作られた。

\begin{figure}[H]
  \centering
  \includegraphics[width=\linewidth]{./src/adder16/adder16_print.png}
  \caption{adder16の回路}
\end{figure}

ロジックエレメント数は19だった。

回路全体の遅延時間は,以下のようになった。

\begin{figure}[H]
  \centering
  \includegraphics[width=\linewidth]{./src/adder16/adder16Timing.png}
  \caption{adder16の遅延時間}
\end{figure}

\subsection{考察}
\subsubsection{回路のHDL記述}
コード\ref{adder16.v}では,16bitの入出力を定義して,assign句で出力に演算結果を割り当てている。
17bit連接に対して足し算の結果を割り当てているため,結果が16bitに収まらない場合,coutの値は1になる。

コード\ref{testadder16.v}ではテストベンチとして,入力をすべてゼロに初期化し,10nsごとにxを100づつ加算し,5nsごとにyを300づつ加算している。

\subsubsection{機能レベルシュミレーション}
シュミレーションの結果は仕様を満たす動作だった。
キャプチャの範囲の外で,coutに桁上げが出力されていることも確認できた。

\subsubsection{論理合成}
今回の合成結果では,一桁ごとに桁上りを含めた3入力の全加算器で加算し,直列につなげていた。
また,遅延時間は入出力の前後に集中していた。


\section{実験3}
\section{実験3}
\subsection{実験の目的,概要}
本実験では、print_B.binにコンパイルされるC言語の記述を、print_B_while.binを生成するように書き換える。
また、作成したコードは実際にコンパイルし、正しいデータと比較する。

この実験で、FPGA上で文字を表示するためのプログラムの書き方、MIPS向けクロスコンパイルの方法を確認することを目的とする。

\subsection{実験方法}
print\_B.cを編集して,以下のようなファイルmy\_print\_B\_while.cを作成した。
\lstinputlisting[caption=my\_print\_B\_while.c,label=myprintBwhile.c]{./src/obj3/my_print_B_while.c}

以下のコマンドで,MIPS用にクロスコンパイルして,生成されたバイナリからメモリイメージファイルを作成した。

\begin{lstlisting}[caption={クロスコンパイル},label={クロスコンパイル}]
  $ cross_compile.sh my_print_B_while.c
  $ bin2v my_print_B_while.bin
\end{lstlisting}

\subsection{実験結果}
コンパイルの結果,このようなメモリイメージファイルが生成された。
\lstinputlisting[caption=rom8x1024.mif,label=rom8x1024.mif3]{src/obj3/rom8x1024.mif}

内容は,実験2-1で使用したソースコード\ref{rom8x1024.mif2-1}と同じだった。

\subsection{考察}


\section{実験課題1}

\subsection{実験の目的,概要}
1桁のBCDコードを出力するBCDカウンタを設計し,それをもとに2桁のBCDコード出力するBCDカウンタを階層設計によって設計する。
設計した2つの回路で,それぞれ機能レベルシュミレーションと論理合成を実行し,入出力の正しさと合成による結果を確認する。

ICE計算機上の,ModelSim,Quartusを使用する。

この実験で,簡単な順序回路の設計方法と,それをもとにした階層設計の方法を実践し,習得することを目的にする。

\subsection{実験方法}
\subsection*{課題1-1}
\subsubsection{回路のHDL記述}
以下のような回路記述をbcd1.v,テストベンチをtest\_bcd1.vとして作成した。
\lstinputlisting[caption=bcd1.v,label=bcd1.v]{./src/bcd/bcd1.v}
\lstinputlisting[caption=test\_bcd1.v,label=testbcd1.v]{./src/bcd/test_bcd1.v}

\subsubsection{機能レベルシュミレーション}
作成したテストベンチをもとに,ModelSimで信号波形を出力した。
入出力の値が仕様通りの真理値表と一致することを確認した。

\subsubsection{論理合成}
以下の2ファイルを作成し,配置した。
\lstinputlisting[caption=bcd1.qpf,label=bcd1.qpf]{./src/bcd/bcd1.qpf}
\lstinputlisting[caption=bcd1.qsf,label=bcd1.qsf]{./src/bcd/bcd1.qsf}

作成した回路記述をQuartusでコンパイルし,論理合成とレイアウトを行した。
回路構成やロジックエレメント数,遅延時間について確認した。

\subsection*{課題1-2}
以下のような回路記述をbcd2.v,テストベンチをtest\_bcd2.vとして作成した
\lstinputlisting[caption=bcd2.v,label=bcd2.v]{./src/bcd/bcd2.v}
\lstinputlisting[caption=test\_bcd2.v,label=testbcd2.v]{./src/bcd/test_bcd2.v}

\subsubsection{機能レベルシュミレーション}
作成したテストベンチをもとに,ModelSimで信号波形を出力した。
入出力の値が仕様通りの真理値表と一致することを確認した。

\subsubsection{論理合成}
以下の2ファイルを作成し,配置した。
\lstinputlisting[caption=bcd2.qpf,label=bcd2.qpf]{./src/bcd/bcd2.qpf}
\lstinputlisting[caption=bcd2.qsf,label=bcd2.qsf]{./src/bcd/bcd2.qsf}

作成した回路記述をQuartusでコンパイルし,論理合成とレイアウトを行った。
回路構成やロジックエレメント数,遅延時間について確認した。

\subsection{実験結果}
\subsection*{課題1-1}
\subsubsection{機能レベルシュミレーション}
ModelSimで波形を作成した結果,以下のような波形になった。

\begin{figure}[H]
  \centering
  \includegraphics[width=\linewidth]{./src/bcd/bcd1wave.png}
  \caption{bcd1の波形}
\end{figure}

bcd1\_outの値は期待通り10進1桁bcdカウンタとして機能している。

\subsubsection{論理合成}
論理合成の結果,以下のような回路が作られた。

\begin{figure}[H]
  \centering
  \includegraphics[width=\linewidth]{./src/bcd/bcd1surc.png}
  \caption{bcd1の回路}
\end{figure}

ロジックエレメント数は5だった。

回路全体の遅延時間は,以下のようになった。

\begin{figure}[H]
  \centering
  \includegraphics[width=\linewidth]{./src/bcd/bcd1timing.png}
  \caption{bcd1の遅延時間}
\end{figure}

\subsection*{課題1-2}
\subsubsection{機能レベルシュミレーション}
ModelSimで波形を作成した結果,以下のような波形になった。

\begin{figure}[H]
  \centering
  \includegraphics[width=\linewidth]{./src/bcd/bcd2wave.png}
  \caption{bcd2の波形}
  \label{bcd2の波形}
\end{figure}

bcd1\_outの値は期待通り10進2桁bcdカウンタとして機能している。

\subsubsection{論理合成}
論理合成の結果,以下のような回路が作られた。

\begin{figure}[H]
  \centering
  \includegraphics[width=\linewidth]{./src/bcd/bcd2surc.png}
  \caption{bcd2の回路}
  \label{bcd2の回路0}
\end{figure}

bcd1モジュールの他に以下のゲート構成があった。

\begin{figure}[H]
  \centering
  \includegraphics[width=\linewidth]{./src/bcd/bcd2-2x.png}
  \caption{xとbcd1a[2]}
  \label{bcd2の回路1}
\end{figure}

\begin{figure}[H]
  \centering
  \includegraphics[width=\linewidth]{./src/bcd/bcd2031.png}
  \caption{bcd2[3,0,1]}
  \label{bcd2の回路2}
\end{figure}

ロジックエレメント数は11だった。

回路全体の遅延時間は,以下のようになった。

\begin{figure}[H]
  \centering
  \includegraphics[width=\linewidth]{./src/bcd/bcd2timing.png}
  \caption{bcd2の遅延時間}
\end{figure}

\subsection{考察}
\subsection*{課題1-1}
\subsubsection{回路のHDL記述}
コード\ref{bcd1.v}ではクロックの立ち上がりで,x=1のときに限ってインクリメントしている。
また,レジスタの値が9以上ならば,ゼロを代入している。

リセット信号は,立ち下がりでレジスタにゼロを代入している。

テストベンチ(コード\ref{testbcd1.v})では,5nsごとのクロック反転と,15nsごとのxの反転を実行している。

また,開始20ns後から40ns後までリセット信号を0にしている。

\subsubsection{機能レベルシュミレーション}
1桁のbcdカウンタとして,クロックとxの値によって0~9までの値をとっていた。

\subsubsection{論理合成}
回路では,4桁分のレジスタ用フリップフロップと,各桁での論理演算回路が4桁分あることがわかる。

\subsection*{課題1-2}
\subsubsection{回路のHDL記述}
コード\ref{bcd2.v}では,bcd1のモジュールを利用して2桁分のBCDカウンタを実現している。
具体的には,1桁目が9で桁上りするときに限って,2桁目のBCDカウンタにxを入力している。
また,クロック信号は2つのモジュールに同じ信号を与えている。

リセット信号は,立ち下がりで2つのbcd1モジュールにリセット信号を入力している。

テストベンチ(コード\ref{testbcd2.v})では,5nsごとのクロック反転と,15nsごとのxの反転を実行している。

また,開始20ns後から40ns後までリセット信号を0にしている。

\subsubsection{機能レベルシュミレーション}
図\ref{bcd2の波形}より,4bitを10進1桁分として,2桁分のBCDカウンタが動作している。

\subsubsection{論理合成}
合成された回路では,1桁目の2bit目とxをとって演算した結果を,他のbitと合成して,2桁目のxとして入力していることがわかる。
この部分が,$(bcd1a\_out == 1001) \land x$の結果として反映されていることが分かった。
ほとんど設計したとおりにモジュールが配置されている。

このとき,bcd1モジュールをインラインで展開した場合,回路の構成が異なる可能性がある。
今回の結果より,モジュールの中身については各モジュールで最適化したのと同じ最適化が施されている。
よって,階層設計による回路よりも,インラインで記述したほうがロジックエレメント数や,遅延時間が改善する可能性がある,と予想される。
今回の実験ではわからなかったが,もしそのような差異があるならば,ボード規模などの制限によっては,記述しやすい階層設計と,インラインの記述を使い分ける必要があると予想した。\\

この実験で,順序回路の階層設計の手法と,階層化によって論理合成がどのように行われるかについて理解できた。


\section{実験課題2}

\subsection{実験の目的,概要}
本実験では,0011および0010が入力されるたびに,1を出力する系列検出回路を設計する。
このとき,状態器械による順序回路記述で設計する。
作成した回路は,機能レベルシュミレーションと論理合成を行い,結果を確認する。

ICE計算機上の,ModelSim,Quartusを使用する。

この実験で,オートマトンの設計を回路にどう反映させるかの方法について,習得することを目的にする。

\subsection{実験方法}
\subsubsection{回路のHDL記述}
以下のような回路記述をm.v,テストベンチをtest\_m.vとして作成する。
\lstinputlisting[caption=m.v,label=m.v]{./src/m/m.v}
\lstinputlisting[caption=test\_m.v,label=testm.v]{./src/m/test_m.v}

\subsubsection{機能レベルシュミレーション}
作成したテストベンチをもとに,ModelSimで信号波形を出力する。
入出力の値が仕様通りの真理値表と一致することを確認する。

\subsubsection{論理合成}
以下の2ファイルを作成し,配置する。
\lstinputlisting[caption=m.qpf,label=m.qpf]{./src/m/m.qpf}
\lstinputlisting[caption=m.qsf,label=m.qsf]{./src/m/m.qsf}

作成した回路記述をQuartusでコンパイルし,論理合成とレイアウトを行う。回路構成やロジックエレメント数,遅延時間について確認する。

\subsection{実験結果}
\subsubsection{機能レベルシュミレーション}
ModelSimで波形を作成した結果,以下のような波形になった。

\begin{figure}[H]
  \centering
  \includegraphics[width=\linewidth]{./src/m/mwave.png}
  \caption{mの波形}
\end{figure}

出力Yは,期待通り2回1を出力していた。

\subsubsection{論理合成}
論理合成の結果,以下のような回路が作られた。

\begin{figure}[H]
  \centering
  \includegraphics[width=\linewidth]{./src/m/mprint.png}
  \caption{mの回路}
  \label{mの回路}
\end{figure}

ロジックエレメント数は5だった。

回路全体の遅延時間は,以下のようになった。

\begin{figure}[H]
  \centering
  \includegraphics[width=\linewidth]{./src/m/mtiming.png}
  \caption{mの遅延時間}
\end{figure}

\subsection{考察}
\subsubsection{回路のHDL記述}
図\ref{設計したオートマトン}のような出力付きのオートマトンをもとにして,回路記述を行った。
レジスタに状態を記憶し,入力に応じて状態を遷移,出力する。

今回は,0010011のような系列に2回検出を行うオートマトンを設計した。

\begin{figure}[H]
  \centering
  \includegraphics[width=5cm]{./src/m/mautomaton.png}
  \caption{設計したオートマトン}
  \label{設計したオートマトン}
\end{figure}
\subsubsection{機能レベルシュミレーション}
テストベンチでは,これまでの順序回路と同様な設定の後に,1→0→0→1→0→0→1→1という系列を10nsおきに入力している。
最初に0010が,次に0011が検出されることが期待されて,実際に2回検出されている。

\subsubsection{論理合成}
コード\ref{m.v}では,4状態のオートマトンと出力レジスタが宣言されているが,図\ref{mの回路}では,合成によって3つの状態と出力,論理演算のみで構成されていることがわかった。
状態として持っていた情報を,現在の状態と入力の演算として再定義することで,状態数を削減しているのだと予想する。\\

この実験で,オートマトンの設計を順序回路として記述する方法がわかった。
また,そのような回路は合成によって状態数が減ることがある,ということも確認できた。


\end{document}
