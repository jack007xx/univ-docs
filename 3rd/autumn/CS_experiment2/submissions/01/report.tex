\documentclass[a4paper,15pt]{jsarticle}

% 数式
\usepackage{amsmath,amsfonts}
\usepackage{bm}
% 画像
\usepackage[dvipdfmx]{graphicx}

\usepackage{listingsutf8,jlisting} %日本語のコメントアウトをする場合jlistingが必要
%ここからソースコードの表示に関する設定
\lstset{
  basicstyle={\ttfamily},
  identifierstyle={\small},
  commentstyle={\smallitshape},
  keywordstyle={\small\bfseries},
  ndkeywordstyle={\small},
  stringstyle={\small\ttfamily},
  frame={tb},
  breaklines=true,
  columns=[l]{fullflexible},
  numbers=left,
  xrightmargin=0zw,
  xleftmargin=3zw,
  numberstyle={\scriptsize},
  stepnumber=1,
  numbersep=1zw,
  lineskip=-0.5ex
}

\begin{document}

\title{コンピュータ科学実験レポート}
\author{坪井正太郎(101830245)}
\date{\today}
\maketitle
\section*{概要}

\section*{はじめに}
各実験で,シュミレータや論理合成のソフトウェアを使うために,以下の設定を行う。
端末を終了した場合,再度sourceコマンドを実行する。
\begin{lstlisting}[caption={設定の読み込み},label={設定の読み込み}]
  $ ln -s /pub1/jikken/eda3/cadsetup.bash.altera ~/
  $ source ~/cadsetup.bash.altera
\end{lstlisting}

\section*{各実験}

\section{実験1}
\subsection{実験の目的,概要}
本実験では,2入力1出力のセレクタ回路を設計し,シュミレーションや論理合成を行って,FPGAボードでの動作を確認する。
HDLによる記述,シュミレータの使い方,回路をFPGA上で動かすための方法を確認する。

入力はD0,D1(1bitデータ),S1(1bitセレクト信号)。
出力はY(1bit)。
セレクタ信号が0ならばD0のデータを,1ならばD1のデータを出力する。

ICE計算機上の,ModelSim,Quartus,FPGAボードはDE10-Liteを使用する。

\subsection{実験方法}
\subsubsection{HDLでの回路記述}
回路記述をmux.v,テストベンチをtest\_mux.vとして作成する。

\subsubsection{機能レベルシュミレーション}
作成したテストベンチをもとに,ModelSimでの入出力の値が仕様通りの真理値表と一致することを確認する。

\subsubsection{コンパイル}
以下の2ファイルを作成し,配置する。
\lstinputlisting[caption=mux21.qpf,label=mux21.qpf]{../../src/mux21/mux21.qpf}
\lstinputlisting[caption=mux21.qsf,label=mux21.qsf]{../../src/mux21/mux21.qsf}

作成した回路記述をQuartusでコンパイルし,論理合成とレイアウトを行う。回路構成やロジックエレメント数,遅延時間について確認する。

\subsubsection{FPGAボードでの回路実現}
計算機にFPGAボードを接続し,dmesgコマンドで接続を確認する。
以下のような設定ファイルを作成する。
\lstinputlisting[caption=mux21.cdf ダウンロード設定ファイル,label=mux21.cdf ダウンロード設定ファイル]{../../src/mux21/mux21.cdf}
\ref{mux21.cdf ダウンロード設定ファイル}を配置し,`quartus-pgm mux21.cdf'を実行し,接続したFPGAにダウンロードする。

FPGAが仕様通りに動作するかを確認する。

\subsection{実験結果}
\subsubsection{回路のHDL記述}
以下のような
\subsubsection{機能レベルシュミレーション}
\subsubsection{論理合成}
\subsubsection{FPGAボードでの動作結果}
\subsection{考察}

\section{実験2}
\subsection{実験の目的,概要}
\subsection{実験方法}
\subsection{実験結果}
\subsubsection{回路のHDL記述}
\subsubsection{機能レベルシュミレーション}
\subsubsection{論理合成}
\subsubsection{FPGAボードでの動作結果}
\subsection{考察}

\section{実験3}
\subsection{実験の目的,概要}
\subsection{実験方法}
\subsection{実験結果}
\subsubsection{回路のHDL記述}
\subsubsection{機能レベルシュミレーション}
\subsubsection{論理合成}
\subsubsection{FPGAボードでの動作結果}
\subsection{考察}

\section{実験課題1}
\subsection{実験の目的,概要}
\subsection{実験方法}
\subsection{実験結果}
\subsubsection{回路のHDL記述}
\subsubsection{機能レベルシュミレーション}
\subsubsection{論理合成}
\subsubsection{FPGAボードでの動作結果}
\subsection{考察}

\section{実験課題2}
\subsection{実験の目的,概要}
\subsection{実験方法}
\subsection{実験結果}
\subsubsection{回路のHDL記述}
\subsubsection{機能レベルシュミレーション}
\subsubsection{論理合成}
\subsubsection{FPGAボードでの動作結果}
\subsection{考察}


\end{document}
