\documentclass[a4paper,10pt]{jsarticle}

% 数式
\usepackage{amsmath,amsfonts}
\usepackage{bm}
% 画像
\usepackage[dvipdfmx]{graphicx}

\usepackage{listingsutf8,jlisting} %日本語のコメントアウトをする場合jlistingが必要
%ここからソースコードの表示に関する設定
\lstset{
  basicstyle={\ttfamily},
  identifierstyle={\small},
  commentstyle={\smallitshape},
  keywordstyle={\small\bfseries},
  ndkeywordstyle={\small},
  stringstyle={\small\ttfamily},
  frame={tb},
  breaklines=true,
  columns=[l]{fullflexible},
  numbers=left,
  xrightmargin=0zw,
  xleftmargin=3zw,
  numberstyle={\scriptsize},
  stepnumber=1,
  numbersep=1zw,
  lineskip=-0.5ex
}

\begin{document}

\title{最適化課題レポート}
\author{坪井正太郎(101830245)}
\date{\today}
\maketitle
\section{}
各色を$c\in \{c_1,c_2,c_3,c_4\}$とする。

多色で塗られたノードの数=全体で使われた色の総数-総ノード数である。

制約条件を
\[\forall i \in V, p_{ic_1}+p_{ic_2}+p_{ic_3}+p_{ic_4}\geq  1\]
\[\forall (i,j)\in E, p_{ic_1}+p_{jc_1}\leq 1,p_{ic_2}+p_{jc_2}\leq 1,p_{ic_3}+p_{jc_3}\leq 1,p_{ic_4}+p_{jc_4}\leq 1,\]
として、目的関数を
\[\sum_{i = 1}^{len(V)} (p_{ic_1}+p_{ic_2}+p_{ic_3}+p_{ic_4})-len(V)\rightarrow 最大 \]
とする。

\end{document}
