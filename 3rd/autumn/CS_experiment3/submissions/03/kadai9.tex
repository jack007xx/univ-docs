\section{課題9}
\subsection{課題概要}
以下の機能を実装した。
\begin{itemize}
  \item forward文による、先行宣言
\end{itemize}

\subsection{実装}
var\_decl\_partのあとにforward\_decl\_partを追加した。
その中で、記号表に対する関数(手続き)名の追加を行った。
また、引数の数も記号表に登録する必要があるので、f\_argsの中でグローバル変数gArityをインクリメントした。

今回の実装では、宣言されたプログラムが定義されていない場合、エラーを出力する必要があり、そのためにグローバル変数としてgUndefinedを用意した。
また、関数として宣言されて手続きとして定義された場合(逆も)にもエラーを出力した。

\subsection{pl4の実行}
pl4-test.p, pl4a.pを実行した結果、以下のようになった。
\begin{lstlisting}[caption={pl4の実行結果},label={pl4の実行結果}]
$ ./parser pl4-test.p
$ lli result.ll
$ 6
  8
$ 100

$ ./parser pl4a.p
$ lli result.ll
$ 100
  5050
\end{lstlisting}

\subsection{考察}
llvm IRではグローバルに定義された関数はどの順序でも呼び出すことができる。
したがって、記号表への追加のみでコード生成のプログラムに変更はなかった。

また、入力が10以下のときにフィボナッチ数列を出力するプログラム(pl4-test)と入力までの数の和を出力するプログラム(pl4a)が正しく動作していることが確認できた。
