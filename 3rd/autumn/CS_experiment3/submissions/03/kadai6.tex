\section{課題6}
\subsection{課題概要}
PL-1コンパイラを完成させるために、以下の機能を実装した。
\begin{itemize}
  \item 定義時に手続きの仮引数を定義する
  \item 呼び出し時に手続きに引数を渡す
  \item 呼び出し時に手続きの引数の数をチェックする
\end{itemize}

\subsection{実装}
引数を扱うために、定義時の仮引数をv\_argsという構文要素で、呼び出し時の実引数をargsという構文要素として定義した。
また、引数の個数は上位の構文要素からはわからないので、gArityというグローバル変数によって管理した。
(定義毎、呼び出し毎に0初期化する。)

\subsubsection{仮引数の定義}
仮引数を利用するためには、\%0から始まる引数をllvmの関数定義の中でallocとstoreで仮引数の値を局所変数として持ってくる必要があり、その際レジスタ番号を消費する。
この動作を実現するために、v\_argsの中で引数の数をカウントしておき、抜けたときに一括でコードを生成、消費したレジスタ番号を返すことでそれ以下のレジスタとの整合性をとる。
これらは、llvm.hで void fundecl\_add\_arg(Factor *); int fundecl\_add\_arg\_code(); として定義した関数を用いて実装した。

\subsubsection{呼び出し時の操作、引数の数をチェック}
手続きの呼び出しは文として、statement内の構文要素proc\_call\_statementとして定義した。
記号表に従ってarityの正しさを検証し、argsに逆順で入っているfactorを取り出し、call命令のコードに渡した。

arityのチェックは、v\_argsのときと同様にgArityを利用して記号との比較で実現した。

\subsection{実行結果}
\subsubsection{pl1a.pの実行}
pl1a.pを実行したところ、以下のような結果となった。
\begin{lstlisting}[caption={pl1a.pの実行コマンド、結果},label={pl1a.pの実行コマンド、結果}]
$ ./parser pl1a.p
$ lli result.ll
$ 5
  120
$ lli result.ll
$ 8
  40320
\end{lstlisting}

\subsubsection{エラーのあるプログラムの実行}
pl1a.pの9行目の手続き呼び出しの際に、存在しない変数を引数として渡してコンパイルした。
\begin{lstlisting}[caption={ミスのあるpl1a.pの実行結果},label={ミスのあるpl1a.pの実行結果}]
$ ./parser pl1a.p
  not decleared yet
  line: 9
  )
  too much procedure argments
  line: 9
  )
\end{lstlisting}

\subsection{考察}
構文要素をまたいで下位の要素がどれだけ連続しているか知る必要があるため、グローバル変数を使って実装した。
また、今回の実装では変数周りのエラー(引数の数も)についてコンパイル時にyyerrorを呼び出した。
これまでの実装ではコンパイル時にエラー(記号表に存在するかなど)をチェックせず、セグフォや実行時エラーを出していたため、コードが書きやすくなった。

実行結果は、階乗を実行するプログラムが正しく動作した。
また、エラー文のも正しいものが出力されていることが確認できた。
