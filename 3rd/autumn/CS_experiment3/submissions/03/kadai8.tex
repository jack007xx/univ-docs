\section{課題7}
\subsection{課題概要}
PL-3コンパイラを完成させるために、以下の機能を実装した。
\begin{itemize}
  \item 配列の定義
  \item 配列の参照
  \item 配列への代入
\end{itemize}

\subsection{実装}
記号表に配列用の新たな種別を追加し、Factor構造体にサイズを持たせるための新たなメンバを追加した。
これにより、通常の命令生成と同じようにallocaやglobalに渡すことができる。

\subsubsection{配列の定義}
id\_listの内部を新たな構文要素id\_declで入れ替え、その中で変数と配列の定義を行った。
表現するデータ構造はほぼ一緒で、メンバの内容だけが違うので、変数の定義とほぼ同様の処理である。

配列のサイズは、最大インデックス-オフセット+1で算出した。

\subsubsection{配列の参照、代入}
var\_name中に、配列の参照部分を作成し、その中でgetelementptr命令を呼び出した。
getelementptrで1次元の配列の要素のポインタを取り出すとき、getelementptr inbounds [size x i32], [size x i32]* array, i32 0, i32 index という呼び出しを行う。
このような命令を生成するコマンドをllvm.h内で宣言した。

\subsection{プログラムの実行結果}
\subsubsection{pl3の実行結果}
pl3を実行した結果、以下のようになった。
\begin{lstlisting}[caption={pl3.pの実行結果},label={pl3.pの実行結果}]
$ ./parser pl3a.p
$ lli result.ll
$ 5
$ 1
$ 2
$ 3
$ 4
$ 5
  5
  4
  3
  2
  1

$ ./parser pl3a.p
$ lli result.ll
$ 20
  2
  3
  5
  7
  11
  13
  17
  19
\end{lstlisting}

\subsection{考察}
getelementptrの1つ目のオペランドに指定されているi32 0は、配列それ自体を指すオフセットである。
また、getelementptrにinboundsキーワードを指定することで、範囲外へのアクセスに対してpoison valueを返す。

pl3の実行では、入力した数列を逆順に出力するプログラム(a)と、入力以下の素数を出力するプログラム(b)が正しく動作していることが確認できた。
