\section{課題10}
\subsection{課題概要}
最適化の機能として、以下の機能を追加した。
\begin{itemize}
  \item コンパイル時の定数計算
  \item 除算の高速化
\end{itemize}

\subsection{実装}
optimize/optimize.hで最適化用のプログラムを実装した。
まず、基本ブロックごとの最適化を行うために、関数列→コード列→基本ブロックの順に分解する関数の中で、基本ブロックの開始と終了ポインタをとる関数として最適化関数を定義した。

最適化された命令列を生成するためにllvm.h内で任意の文字列をコードとして埋め込むことの出来る疑似命令Freeを実装した。
通常変数名となる部分に命令文字列を格納することで、その文字列を展開する。

また、llvm IRではレジスタが番号でつけられていると、番号の間が飛んだ場合にエラーが発生する。
そのため、全てのレジスタ番号にプレフィックスをつけることでレジスタ名による指定に切り替えた。

\subsubsection{コンパイル時の定数計算}
四則演算命令のうち、オペランドが2つとも定数である場合、そのコードを削除、格納先のレジスタを定数に書き換え、値として命令の演算結果を入力した。
連続して行うことで、計算可能なコード列については全て置き換えることができる。

\subsubsection{除算の高速化}
除算の高速化として「2のべき乗による割り算のシフト化」「3で割る処理を乗算で実行」の2種類を実装した。

\subsubsection{2のべき乗による割り算のシフト化}
2のべき乗による割り算を、右シフト命令として置き換えた。
今回のプログラムは全て符号付き整数なので、符号拡張を行うashr命令によって右シフトを実行した。

また、負の数を割る際の丸めを正しく実行するために、負の数を割る場合にのみ符号ビットをあらかじめ加算した。
この処理は、31ビット分ashrで右シフトすることで0または-1を取り出すことで実現した。

\subsubsection{3で割る処理を乗算で実行}
2のべき乗を3で割った巨大な整数を掛け合わせることで、べき乗分右シフトすると3で割った数を出力できる。
今回は2の32乗を3で割り、上に切り上げた数1431655766を使った。
2で割ったときと同様に、負の数の丸め分を補うために、1を加算する処理も行った。
この処理は、符号ビットを結果から減算することで実装した。

また、2の32乗を掛けた数は明らかにi32には収まらない。
そのため、型変換拡張命令sextと縮小命令truncを前後につけた。

\subsection{テストプログラムの実行}
以下のようなプログラムを用意して、コンパイル結果を確認、timeコマンドでベンチマークをとった。
\begin{lstlisting}[caption={テストプログラム},label={テストプログラム}]
program bench;
  var x, n, nn;
  begin
     for n := 1 to 1000000 do
      for nn := 1 to 5000 do
        begin
            x := n div (2 + 1);
            x := n div 2;
        end
  end.
\end{lstlisting}

\subsubsection{コンパイル結果}
ソースコード\ref{テストプログラム}をコンパイルした結果、以下のプログラムが生成された。
for.doブロック内において、定数のコンパイル時計算(2+1→3)と、2と3の割り算の最適化が行われていることが確認できる。

\begin{lstlisting}[caption={コンパイル結果},label={コンパイル結果}]
  @.str.w = private unnamed_addr constant [4 x i8] c"%d\0A\00", align 1
  @.str.r = private unnamed_addr constant [3 x i8] c"%d\00", align 1
  declare i32 @printf(i8*, ...)
  declare i32 @__isoc99_scanf(i8 *, ...)
  
  @x = common global i32 0, align 4
  @n = common global i32 0, align 4
  @nn = common global i32 0, align 4
  
  define i32 @main(){
    store i32 1, i32* @n, align 4
    br label %o1
  
  ; <for.loop>:
  o1:
    %o2 = load i32, i32* @n, align 4
    %o3 = icmp sle i32 %o2, 1000000
    br i1 %o3, label %o4, label %o21
  
  ; <for.do>:
  o4:
    store i32 1, i32* @nn, align 4
    br label %o5
  
  ; <for.loop>:
  o5:
    %o6 = load i32, i32* @nn, align 4
    %o7 = icmp sle i32 %o6, 5000
    br i1 %o7, label %o8, label %o17
  
  ; <for.do>:
  o8:
    %o9 = load i32, i32* @n, align 4
  
    ;[OPTIMIZED]
    ;div 3 -> sift hogehoge
    %o9.1 = sext i32 %o9 to i64
    %o9.2 = ashr i64 %o9.1, 31
    %o9.3 = mul i64 %o9.1, 1431655766
    %o9.4 = ashr i64 %o9.3, 32
    %o9.5 = sub i64 %o9.4, %o9.2
    %o11 = trunc i64 %o9.5 to i32
  
    store i32 %o11, i32* @x, align 4
    %o12 = load i32, i32* @n, align 4
  
    ;[OPTIMIZED]
    ;div 2^n -> sift-R n
    %o12.1 = ashr i32 %o12, 31
    %o12.2 = add i32 %o12, %o12.1
    %o13 = ashr i32 %o12.2, 1
  
    store i32 %o13, i32* @x, align 4
    br label %o14
  
  ; <for.increment>:
  o14:
    %o15 = load i32, i32* @nn, align 4
    %o16 = add nsw i32 %o15, 1
    store i32 %o16, i32* @nn, align 4
    br label %o5
  
  ; <for.break>:
  o17:
    br label %o18
  
  ; <for.increment>:
  o18:
    %o19 = load i32, i32* @n, align 4
    %o20 = add nsw i32 %o19, 1
    store i32 %o20, i32* @n, align 4
    br label %o1
  
  ; <for.break>:
  o21:
    ret i32 0
  }
  
  
\end{lstlisting}

\subsubsection{ベンチマーク結果}
最適化の前後で10回づつ計測し、timeコマンドのうちuserのCPU時間をとった。
\begin{table}[H]
  \centering
  \label{ベンチマーク結果}
  \caption{ベンチマーク結果}
  \begin{tabular}{|c|c|c|}
    \hline
           & 最適化前(s) & 最適化後(s) \\ \hline
    1回目  & 9.047       & 9.052       \\ \hline
    2回目  & 9.028       & 9.123       \\ \hline
    3回目  & 9.102       & 9.074       \\ \hline
    4回目  & 9.020       & 9.068       \\ \hline
    5回目  & 9.021       & 9.154       \\ \hline
    6回目  & 9.009       & 9.102       \\ \hline
    7回目  & 9.023       & 9.124       \\ \hline
    8回目  & 9.026       & 9.071       \\ \hline
    9回目  & 8.968       & 9.043       \\ \hline
    10回目 & 9.050       & 9.448       \\ \hline
    平均   & 9.294       & 9.126       \\ \hline
  \end{tabular}
\end{table}
\subsection{考察}
今回の最適化では、定数の計算に関わる部分に関してのみ基本的な最適化を行った。
一般に、プロセッサレベルでは除算はその他の四則演算より大幅に遅いことから、その部分について特に重点的に最適化を行った。

\subsubsection{3の除算}
割られる数をq、結果をpとすると、以下の等式が成り立つ。
\[p=q\left(\frac{2^{32}}{3}\right)\left(\frac{1}{2^{32}}\right)\]
ここで、\(\frac{2^{32}}{3}=1431655765.33\)であり、小数点以下については\(\frac{1}{2^{32}}\)を掛けたときに必ず切り捨てられる。
そのため、q,pを整数に制限すると
\[p=q\left(1431655766\right)\left(\frac{1}{2^{32}}\right)\]
としてもよい。
負の数の表現を考慮して結果には符号ビットを足す必要があるが、3の除算についてはこれによって掛け算と右シフトで表現できる。

さらに、同様の計算を他の数に対しても適用できるため、ある程度小さな数については、3以外の除算に対しても同じ最適化ができる。
これによって、2のべき乗との組み合わせで様々な数の除算が乗算に変更できる。

\subsubsection{ベンチマークに対する考察}
ベンチマークをとったところ、最適化後のほうが平均で0.17秒(2\%)ほど遅い結果となった。
これは、i32←→i64の型変換によるオーバーヘッドの影響であると予想し、追実験を行った。
コンパイル結果のソースコード\ref{コンパイル結果}のうち、型変換に関する
37,42行目を削除し、実行にエラーが起きないように削除されたレジスタ部分を定数で書き換えて再度10回実行した。

結果は以下のようになり、0.313秒(3.3\%)ほど早くなった。
予想通り、型変換のオーバーヘッドが原因であることが分かった。
\begin{table}[H]
  \centering
  \label{追実験結果}
  \caption{追実験結果}
  \begin{tabular}{|c|c|c|}
    \hline
           & 最適化前(s) & 型変換なし最適化(s) \\ \hline
    1回目  & 9.047       & 8.968               \\ \hline
    2回目  & 9.028       & 9.052               \\ \hline
    3回目  & 9.102       & 8.992               \\ \hline
    4回目  & 9.020       & 9.017               \\ \hline
    5回目  & 9.021       & 9.019               \\ \hline
    6回目  & 9.009       & 9.051               \\ \hline
    7回目  & 9.023       & 9.043               \\ \hline
    8回目  & 9.026       & 8.900               \\ \hline
    9回目  & 8.968       & 8.846               \\ \hline
    10回目 & 9.050       & 8.928               \\ \hline
    平均   & 9.294       & 8.981               \\ \hline
  \end{tabular}
\end{table}

このことから、3の除算最適化はLLVM IRにおいては効果が打ち消され、レジスタの幅が利用する型ごとに決定されるLLVMとの相性が悪いと考えた。
逆にレジスタが、利用する整数型の倍以上の幅を持っている場合や、整数型のうち、下位半分のビットしか使わない場合には一定の効果があると言える。
