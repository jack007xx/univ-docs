\section{課題7}
\subsection{課題概要}
PL-2コンパイラを完成させるために、以下の機能を実装した。
\begin{itemize}
  \item 関数の定義
  \item 関数の戻り値の生成
  \item 定義時、呼び出し時の引数のチェック(手続きと一緒なので実装は省略)
\end{itemize}

\subsection{実装}
関数の定義は、LLVMにおいて関数と手続きの区別がないことから構文要素func\_declを追加し、その中でproc\_declと同じような操作を行った。

また、関数の戻り値は項として必ず利用され、文(statement)として扱われることはないとした。
そのため、関数呼び出しは新たな構文要素func\_call\_factorを構文要素factorの中で定義した。

\subsubsection{関数の定義}
関数の定義は、LLVMにおいて関数と手続きの区別がないことから構文要素func\_declを追加し、その中でproc\_declと同じような操作を行った。

\subsubsection{関数の戻り値の生成}
関数は戻り値を持つため、そのための領域を関数の先頭でallocaしておき、これをグローバル変数として保持した。
変数名として、現在定義中の関数名と同じ名前のものが渡された場合には、保持しておいた戻り値領域を利用する。

\subsection{実行結果}
実行結果は以下のようになった。

\subsubsection{pl2の実行結果}
\begin{lstlisting}[caption={pl2a.p, pl2b.pの実行結果 },label={pl2a.p, pl2b.pの実行結果}]
$ ./parser pl2a.p
$ lli result.ll
$ 5
  120
$ ./parser pl2b.p
$ lli result.ll
$ 2 
$ 4
  16
\end{lstlisting}

\subsubsection{関数を手続きとして呼び出したときの実行結果}
pl2a.pの12行目に手続きとしてfact(n);を挿入し、実行した。
\begin{lstlisting}[caption={関数を手続きとして呼び出す},label={関数を手続きとして呼び出す}]
$ ./parser pl2a.p
  not decleared as procedure
  line: 12
  )
\end{lstlisting}

\subsection{考察}
今回の実装では、関数は文として使われないという仕様を決めた。
実際に、実行結果では関数が手続きと同様に呼び出されたときにエラーを出力している。

また、通常のpl2の実行結果から、階乗を行うプログラム(a)と、2つの数のべき乗を求めるプログラム(b)が正しく動作していることが確認できる。
