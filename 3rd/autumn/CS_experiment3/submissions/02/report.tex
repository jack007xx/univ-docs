\documentclass[a4paper,10pt]{jsarticle}

% 数式
\usepackage{amsmath,amsfonts}
\usepackage{bm}
% 画像
\usepackage[dvipdfmx]{graphicx}
\usepackage{here}

\usepackage{listingsutf8,jlisting} %日本語のコメントアウトをする場合jlistingが必要
%ここからソースコードの表示に関する設定
\lstset{
  basicstyle={\ttfamily},
  identifierstyle={\small},
  commentstyle={\smallitshape},
  keywordstyle={\small\bfseries},
  ndkeywordstyle={\small},
  stringstyle={\small\ttfamily},
  frame={tb},
  breaklines=true,
  columns=[l]{fullflexible},
  numbers=left,
  xrightmargin=0zw,
  xleftmargin=3zw,
  numberstyle={\scriptsize},
  stepnumber=1,
  numbersep=1zw,
  lineskip=-0.5ex
}

\begin{document}

\title{実験レポート2}
\author{坪井正太郎(101830245)}
\date{\today}
\maketitle
\section{はじめに}
\subsection{コード規約}
\subsubsection*{ケース}
\begin{itemize}
  \item 型名
        \begin{itemize}
          \item UpperCamelCase
        \end{itemize}
  \item 構造体名、メンバ名、変数名
        \begin{itemize}
          \item lowerCamelCase
        \end{itemize}
  \item 関数名
        \begin{itemize}
          \item snake\_case
        \end{itemize}
\end{itemize}

\subsubsection*{変数のプレフィックス}
\begin{itemize}
  \item グローバル変数:g
        \begin{itemize}
          \item gVar
        \end{itemize}
  \item ローカル変数:t
        \begin{itemize}
          \item tVar
        \end{itemize}
  \item 仮引数:a
        \begin{itemize}
          \item aVar
        \end{itemize}
\end{itemize}

\subsubsection*{コメント}
型、関数の説明はヘッダにのみ記載、cファイルには実装に関するコメントのみ。
プライベート関数は明記して実装にコメント。

\section{課題4}


\newpage
\section{課題5}
\subsection{課題概要}
PL-0をコンパイルするために、以下のような機能をコンパイラに追加した。
\begin{itemize}
  \item 比較演算
  \item if,while,forなどの制御文
  \item 手続きの呼び出し
  \item read,write手続き
\end{itemize}

本来であれば、変数の扱いは課題5の内容だが、課題4で既に実装、説明してあるのでその部分は省略する。

\subsection{実装}

\subsubsection{比較演算}
llvm.hで新たな命令icmpを定義した。
icmp命令は比較演算の種類を指定する必要があるので、Cmptype列挙体型を用意してコード生成時に引数として渡すこととした。
それに伴って、parser.yでcode\_create命令を呼び出している部分を変更した。

具体的な生成は、parser.yで以下のように行った。

\begin{lstlisting}[caption={condition},label={condition}]
  condition
  : expression EQ expression
  {
          // 四則演算と大体一緒
          Factor *tArg2 = factor_pop();
          Factor *tArg1 = factor_pop();
          Factor *tRetval = factor_push("", gRegnum++, LOCAL_VAR);

          code_add(code_create(Icmp, tArg1, tArg2, tRetval, EQUAL));
  }
!!その他の比較演算も同様に定義。
\end{lstlisting}

\subsubsection{if,while,forなどの制御文}
まず、ラベルを管理しやすくするために、symtabで定義したScopeを変更し、factorでラベルを扱えるようにした。
LLVMにはLabel命令はないが、llvm.h,llvm.cを変更し、コード生成時には命令としてLabelを指定し、ラベルとしてfactorを渡すことで生成させた。

これによって、バックパッチの実現がfactorのスタックを通して行える。
以下に示すように、後から番号を付けたいラベルをpushしておき、for文の内部の文が読まれてからpopして*Factor.valに適切な値を代入する。
br系命令のラベルと、バックパッチされるlabel文は同じポインタを指しているので、レジスタ割当が未確定のラベルに対しても分岐できるようになる。

\begin{lstlisting}[caption={バックパッチの方法},label={バックパッチの方法}]
hoge_statement
  : hoge HUGA hoge
  {
    Factor *undef_label = factor_push("label_name", 0, LABEL);
    // popしない!

    code_add(code_create(BrCond, undef_label, hoge, huga_condition, 0));
  }
   statement // この中でもレジスタ番号は消費される
  {
    Factor *back_patch_label = factor_pop();
    back_patch_label->val = gRegnum++;

    code_add(code_create(BrUncond, back_patch_label, NULL, NULL, 0));
    code_add(code_create(Label, back_patch_label, NULL, NULL, 0));
  }
  ;
\end{lstlisting}

各制御構文のうち、forに関してのみparser.yでの生成箇所と、生成されたコードについての説明を行う。

clangを使うことで確かめたLLVM IRは以下のような命令列であった。
これに加えて、expression TO expressionでpushされた2つのfactorをpopし、(1)~(2)の間で使用する。

\begin{lstlisting}[caption={for文のLLVM IR},label={for文のLLVM IR}]
  カウンタ変数に初期値を代入する
LABEL (1)
  カウンタ変数と上限値の比較
  上限以下の場合(2)、上限より大きい場合(4)にジャンプする
LABEL (2)
  for文の中で行う操作
LABEL (3)
  カウンタ変数をインクリメントする
  (1)にジャンプする
LABEL (4)
\end{lstlisting}

実際の実装は以下のようになった。
カウンタレジスタはstatementの前後で同じ値を使うために、statementの前でpushしておき、後ろからpopできるようにした。
ラベルについては説明したように、pushのみ行いpopしたときに実際のレジスタ番号を割り当てた。
これらの操作は、1つのstatementでfactorのスタックが変化しないことが保証されているため、可能である。
また、各制御文の中でもfactorのスタックを文の前後で保存するように操作した。

\begin{lstlisting}[caption={for文の生成箇所},label={for文の生成箇所}]
for_statement
        : FOR IDENT ASSIGN expression TO expression DO
        {
                Row *tRow = symtab_lookup($2);
                Factor *tTo = factor_pop(); // tFromからtToまで
                Factor *tFrom = factor_pop();

                Factor *tCnt = factor_push(tRow->name, tRow->regnum, tRow->type);
                // カウンタ(インクリメントしていく変数)
                //popせずにとっておく

                code_add(code_create(Store, tFrom, tCnt, NULL, 0));

                Factor *tLoop = factor_push("for.loop", gRegnum++, LABEL);
                // ループで戻ってくる場所、あとからpopしてbr命令を書く

                code_add(code_create(BrUncond, tLoop, NULL, NULL, 0));
                code_add(code_create(Label, tLoop, NULL, NULL, 0));

                factor_push("", gRegnum++, LOCAL_VAR);
                Factor *tCntLocal = factor_pop();
                code_add(code_create(Load, tCnt, NULL, tCntLocal, 0)); // カウンタをレジスタに持ってくる

                factor_push("", gRegnum++, LOCAL_VAR);
                Factor *tCond = factor_pop(); // 条件を用意する

                code_add(code_create(Icmp, tCntLocal, tTo, tCond, SLE));

                factor_push("for.do", gRegnum++, LABEL);
                Factor *tDo = factor_pop(); // 条件でブレークしないときにここに飛ぶ

                Factor *tBreak = factor_push("for.break", 0, LABEL);
                // バックパッチであとから値入れたいのでpushしたままにする。

                code_add(code_create(BrCond, tDo, tBreak, tCond, 0));
                code_add(code_create(Label, tDo, NULL, NULL, 0));
        }
          statement
        {
                // この順番で出てくる
                Factor *tBreak = factor_pop();
                Factor *tLoop = factor_pop();
                Factor *tCnt = factor_pop();

                // cntインクリメント部ここから
                factor_push("for.increment", gRegnum++, LABEL);
                Factor *tInc = factor_pop();
                code_add(code_create(BrUncond, tInc, NULL, NULL, 0));
                code_add(code_create(Label, tInc, NULL, NULL, 0));

                factor_push("", gRegnum++, LOCAL_VAR);
                Factor *tCntLocal = factor_pop();

                code_add(code_create(Load, tCnt, NULL, tCntLocal, 0)); // カウンタをレジスタに持ってくる

                factor_push("", 1, CONSTANT);
                Factor *tOne = factor_pop();

                factor_push("", gRegnum++, LOCAL_VAR);
                Factor *tRetval = factor_pop();

                code_add(code_create(Add, tCntLocal, tOne, tRetval, 0));

                code_add(code_create(Store, tRetval, tCnt, NULL, 0));
                // インクリメント部ここまで

                code_add(code_create(BrUncond, tLoop, NULL, NULL, 0));

                tBreak->val = gRegnum++; // バックパッチ
                code_add(code_create(Label, tBreak, NULL, NULL, 0));
        }
        ;
\end{lstlisting}

\subsubsection{手続きの呼び出し}
LLVM IRでの手続き呼び出しはcall命令を使用する。
llvm.c, llvm.hにはcallの定義と生成ルールを実装した。

parser.yでは以下のようにロード命令に近い形で生成した。

\begin{lstlisting}[caption={手続きの呼び出し生成部分},label={手続きの呼び出し生成部分}]
proc_call_name
: IDENT
{
        Row *tRow = symtab_lookup($1);
        factor_push(tRow->name, tRow->regnum, tRow->type);
        Factor *tProc = factor_pop();

        factor_push("", gRegnum++, LOCAL_VAR);//関数の戻り血
        Factor *tRetval = factor_pop();
        code_add(code_create(Call, tProc, NULL, tRetval, 0));
}
;
\end{lstlisting}

\subsubsection{read,write手続き}
clangを利用してscanf,printfがどのように表現されるか確かめたところ、それぞれの関数の宣言と、フォーマット文字列が定義されていた事がわかった。
取り出したものは以下のようになる。

@.str.wは整数に置換される部分(\%d)、改行コード(0A)、終端文字(00)で、@.str.rは整数部と終端文字で構成され、それぞれprintf,scanfで使用する。
呼び出し時はソースコード\ref{read,writeの呼び出し}で示すように、call命令を使用して引数に出力する整数または、読み込まれる変数を渡す。

\begin{lstlisting}[caption={read,writeのための宣言},label={read,writeのための宣言}]
@.str.w = private unnamed_addr constant [4 x i8] c"%d\0A\00", align 1
@.str.r = private unnamed_addr constant [3 x i8] c"%d\00", align 1
declare i32 @printf(i8*, ...)
declare i32 @__isoc99_scanf(i8 *, ...)
\end{lstlisting}

\begin{lstlisting}[caption={read,writeの呼び出し},label={read,writeの呼び出し}]
  call i32 (i8*, ...) @printf(i8* getelementptr inbounds ([4 x i8], [4 x i8]* @.str.w, i64 0, i64 0), i32 「レジスタ」)
  call i32 (i8*, ...) @__isoc99_scanf(i8* getelementptr inbounds ([3 x i8], [3 x i8]* @.str.r, i64 0, i64 0), i32* 「変数」)
\end{lstlisting}


\subsection{pl0a.p~pl0d.pのコンパイルと実行}
以下のコマンドでそれぞれのソースに対してコンパイルを行い、実行した。
出力されない結果はret文を書き換えてステータスコードを参照することで確かめた。

pl0a.pの結果はn=0,sum=55となり、正しい結果が確認できた。
また、b,c,dはそれぞれ、入力以下の素数を大きなものから出力するプログラム、同様に小さなものから出力するプログラム、入力値の階乗を出力するプログラムであることが確認でき、入出力が正しいことが確認できた。

\begin{lstlisting}[caption={pl0a.p~pl0d.pのコンパイルと実行コマンド},label={pl0a.p~pl0d.pのコンパイルと実行コマンド}]
$ make
$ ./parser pl0a.p
$ lli result.ll
$ ./parser pl0b.p
$ lli result.ll
  10
  7
  5
  3
  2
$ ./parser pl0c.p
$ lli result.ll
  10
  2
  3
  5
  7
$ ./parser pl0d.p
$ lli result.ll
  10
  3628800
\end{lstlisting}

\subsection{考察}
課題4の考察で触れたように、factorのスタックはstatementを抜けるときに必ず入ったときと同じ状態になる。
課題5の実装では、factorにラベルを記憶できるようにしたことで、その特徴をバックパッチに活用することができた。
これによって、コード量を削減することができ、可読性が向上したと考える。


\end{document}
