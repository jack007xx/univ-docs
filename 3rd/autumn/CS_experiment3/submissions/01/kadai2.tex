\section{実験課題2}
\subsection{課題概要}
PL-0を受理するような構文解析プログラムを実装する。

PL-0の構文規則を参考に、paser.yを編集してyaccによってy.tab.hとy.tab.cを生成する。
実際に解析するためには、scanner.lの中で構文解析部を呼び出すための編集も行う。

\subsection{実装}
構文の定義に従い、parser.yには構文要素とその定義を追加した。
また、エラー時には字句解析部で用意される行番号とエラー時の字句を出力するよう編集を行った。

\lstinputlisting[caption=parser.y,label=parser.y]{../../prac/kadai2/parser.y}

\begin{lstlisting}[caption={scanner.l},label={scanner.l2}]
  %{
    /* 
     * scanner: scanner for PL-*
     * 
     */
    
    #include <stdio.h>
    #include <string.h>
    
    #define MAXLENGTH 16
    
    #include "y.tab.h"
    
    %}
    %option yylineno
    %%
    
    begin           return SBEGIN;
    div             return DIV;
    
    省略

    %%
    
    main(int argc, char *argv[]) {
        FILE *fp;
        int tok;
        
        if (argc != 2) {
            fprintf(stderr, "usage: %s filename\n", argv[0]);
            exit(1);
        }
    
        if ((fp = fopen(argv[1], "r")) == NULL) {
            fprintf(stderr, "cannot open file: %s\n", argv[1]);
            exit(1);
        }
    
        /*
         * yyin は lex の内部変数であり,入力のファイルポインタを表す.
         */
        yyin = fp;
    
        yyparse();
    }
    
\end{lstlisting}

\subsection{実行結果}
正しいプログラムが渡された場合、出力はなかった。
誤りのあるプログラムpl0a-err.pを入力した場合、出力は以下のようになった。
\begin{lstlisting}[caption={pl0a-err.pの出力},label={pl0a-err.pの出力}]
$ ./parser pl0a-err.p
syntax error
while
6
\end{lstlisting}

\subsection{考察}
parser.yのエラー出力部では、yytext、yylinenoのグローバル変数を参照して構文解析のエラー結果を出力している。

また、実際に構文解析を行うためには、字句解析の内容をparserにわたす必要がある。
そのために、scanner.lではyaccで生成されたy.tab.hをインクルードしてyyparse関数によって構文解析部を呼び出した。
