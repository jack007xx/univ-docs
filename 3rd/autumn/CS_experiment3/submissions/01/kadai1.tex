\section{実験課題1}
\subsection{課題概要}
scanner.lのルール部を更新し、lexによって字句解析のためのCコードを生成する。

\subsection{実装}
scanner.lに、以下のようなルールを追加した。

\begin{lstlisting}[caption={scanner.l(ルールの記述)},label={scanner.l(ルールの記述)}]
  begin           return SBEGIN;
  div             return DIV;
  
  do              return DO;
  else            return ELSE;
  end             return SEND;
  for             return FOR;
  forward         return FORWARD;
  function        return FUNCTION;
  if              return IF;
  procedure       return PROCEDURE;
  program         return PROGRAM;
  read            return READ;
  then            return THEN;
  to              return TO;
  var             return VAR;
  
  while           return WHILE;
  write           return WRITE;
  
  "+"             return PLUS;
  "-"             return MINUS;
  
  "*"             return MULT;
  "="             return EQ;
  "<>"            return NEQ;
  "<"             return LT;
  "<="            return LE;
  ">"             return GT;
  ">="            return GE;
  "("             return LPAREN;
  ")"             return RPAREN;
  "["             return LBRACKET;
  "]"             return RBRACKET;
  ","             return COMMA;
  ";"             return SEMICOLON;
  ":"             return COLON;
  ".."            return INTERVAL;
  
  "."             return PERIOD;
  ":="            return ASSIGN;
\end{lstlisting}

\subsection{実験結果}
\$./scanner ex1.p を実行した結果、以下のような出力となった。

\begin{lstlisting}[caption={./scanner ex1.pの実行結果},label={./scanner ex1.pの実行結果}]
  "program":      11      RESERVE
  "EX1":  38      EX1
  ";":    32      RESERVE
  "var":  15      RESERVE
  "x":    38      x
  ",":    31      RESERVE
  "y":    38      y
  ",":    31      RESERVE
  "z":    38      z
  ";":    32      RESERVE
  "begin":        1       RESERVE
  "x":    38      x
  ":=":   36      RESERVE
  "12":   37      12
  ";":    32      RESERVE
  "y":    38      y
  ":=":   36      RESERVE
  "20":   37      20
  ";":    32      RESERVE
  "z":    38      z
  ":=":   36      RESERVE
  "x":    38      x
  "+":    18      RESERVE
  "y":    38      y
  "end":  5       RESERVE
  ".":    35      RESERVE
\end{lstlisting}

\subsection{考察}
scanner.lでは予約語に対して、symbols.hで定義されている列挙体の、対応する要素を返している。
実際に実験結果では、定義されたものに対応する整数と、予約語であればRESERVE、識別子であればそれが表示される。
