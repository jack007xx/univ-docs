\documentclass[a4paper,10pt]{jsarticle}

% 数式
\usepackage{amsmath,amsfonts}
\usepackage{bm}
% 画像
\usepackage[dvipdfmx]{graphicx}

\usepackage{listingsutf8,jlisting} %日本語のコメントアウトをする場合jlistingが必要
%ここからソースコードの表示に関する設定
\lstset{
  basicstyle={\ttfamily},
  identifierstyle={\small},
  commentstyle={\smallitshape},
  keywordstyle={\small\bfseries},
  ndkeywordstyle={\small},
  stringstyle={\small\ttfamily},
  frame={tb},
  breaklines=true,
  columns=[l]{fullflexible},
  numbers=left,
  xrightmargin=0zw,
  xleftmargin=3zw,
  numberstyle={\scriptsize},
  stepnumber=1,
  numbersep=1zw,
  lineskip=-0.5ex
}

\begin{document}

\title{整体情報処理 第2回レポート課題}
\author{坪井正太郎(101830245)}
\date{\today}
\maketitle
\section{脳幹,小脳,大脳のそれぞれの機能について}
\subsection{脳幹の機能について}
脳幹は間脳,中脳,延髄からなる。
また脳幹は全体で生命の維持に必要な機能や,非随意的な機能を提供する。

生命維持機能として,呼吸や血管活動や,心臓の制御,唾液分泌や嚥下が挙げられる。
これらの機能は,主に延髄が司る。

中脳は,視覚と聴覚の中継点として機能する。
また,瞳孔の収縮や,ピントの調整,音がした方向に目を向ける反射などを司る。

間脳は,大脳との一部の入出力を担当する。
感覚器の入力は,中脳で中継されて間脳を通して大脳に伝えられる。
また,大脳からの出力は,間脳からホルモンを出すことで全身に伝わる。
間脳は,ホルモンを分泌できる脳下垂体を持つ。

\subsection{小脳の機能について}
小脳は,主に運動を制御する。
平衡感覚や,運動の記憶を担当している。
そのため,損傷した場合平衡感覚や,歩行に障害が生じる。

一方で,損傷した機能を他の部位が大小することもできる。

\subsection{大脳の機能について}
大脳は,感覚と運動の中枢として,脳の中でも高次な機能を提供する。
発生の過程として,大脳の機能は様々な段階がある。

部位は大脳基底核,辺縁系,皮質がある。
基底核では運動の調節,具体的には運動を安定化させたり,運動を開始するときに機能する。

辺縁系は,記憶の書き込みや,情動による判断を行う。
感覚器からの入力を視床より早く,直接処理することもある。

皮質では,理性を司る。
より原始的な旧皮質では本能による判断に,状況判断を加える。
発達している新皮質では,情動を理性で優先して判断する機能を担当する。\\

大脳はある機能について,全体で処理しているが,特に働いている部位がある。
大きくは,感覚器からの刺激を受け取る感覚野,筋肉の司令を出す運動野,情報処理や,論理思考のための連合野がある。
ヒトは連合野が最も大きい。

\section{右脳と左脳について}
大脳を半分に分割した右脳と左脳は,それぞれが反対側の感覚と運動を制御する。
また,左脳では言語処理や計算を,右脳では直感や空間把握を司っている。

2つの半球は,脳梁を通してつながっている。
そのため,脳梁が切断されると言語処理と,非言語処理の間で結果が異なるようになる。
また,視野の左右でも処理に差異が出る。

\end{document}
