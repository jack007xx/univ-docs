\documentclass[a4paper,10pt]{jsarticle}

% 数式
\usepackage{amsmath,amsfonts}
\usepackage{bm}
% 画像
\usepackage[dvipdfmx]{graphicx}

\usepackage{listingsutf8,jlisting} %日本語のコメントアウトをする場合jlistingが必要
%ここからソースコードの表示に関する設定
\lstset{
  basicstyle={\ttfamily},
  identifierstyle={\small},
  commentstyle={\smallitshape},
  keywordstyle={\small\bfseries},
  ndkeywordstyle={\small},
  stringstyle={\small\ttfamily},
  frame={tb},
  breaklines=true,
  columns=[l]{fullflexible},
  numbers=left,
  xrightmargin=0zw,
  xleftmargin=3zw,
  numberstyle={\scriptsize},
  stepnumber=1,
  numbersep=1zw,
  lineskip=-0.5ex
}

\begin{document}

\title{眼球運動について}
\author{坪井正太郎(101830245)}
\date{\today}
\maketitle
\section{眼球運動について}
視野のうち、視力が強いのは中心からおよそ1度20分の範囲であり、それより外側の視力は急激に低下する。
そのため、見るものに対して目を動かす必要がある。

\section{眼球運動の種類}
身体の動きによらない、随意的に行う眼球運動には、随従性運動、断続性運動の2つがある。

随従性運動は、視対象が動いているときに見られる運動であり、動いているものを追う動作の眼球運動である。
最大でも、秒間30度程度しか動かず、なめらかな眼球運動である。

断続性運動は、視対象に素早く注視する眼球運動で、最大で秒間400度の視点移動ができる。
例として、本を読んでいる状態が挙げられる、文字を追う動作は断続性運動である。
また、眼球を視点に移動させる間は、知覚が抑制されて、視対象以外の情報が入りにくくなる。

断続性運動は覚醒時では1秒に4~5回、睡眠中にも発生し、REM睡眠という浅い睡眠になる。

\subsection*{固視微動}
視対象に注視している間も、眼球はわずかに動いている。
この運動は不随意運動であり、固視微動という。

固視微動を停止させると、視界が消失する。
これは、視覚系が像のうち時間的変化のみを脳へ送っているからであり、静止した視界での知覚のためには固視微動が必要である。

視覚系が時間的変化のみを脳に送っているのは、意味のない情報を送ることがないようにする、一種のデータ圧縮であるといえる。
これは、断続性運動中の知覚の抑制にも同じ理由であると考察できる。

\end{document}
