\documentclass[a4paper,10pt]{jsarticle}

% 数式
\usepackage{amsmath,amsfonts}
\usepackage{bm}
% 画像
\usepackage[dvipdfmx]{graphicx}

\usepackage{listingsutf8,jlisting} %日本語のコメントアウトをする場合jlistingが必要
%ここからソースコードの表示に関する設定
\lstset{
  basicstyle={\ttfamily},
  identifierstyle={\small},
  commentstyle={\smallitshape},
  keywordstyle={\small\bfseries},
  ndkeywordstyle={\small},
  stringstyle={\small\ttfamily},
  frame={tb},
  breaklines=true,
  columns=[l]{fullflexible},
  numbers=left,
  xrightmargin=0zw,
  xleftmargin=3zw,
  numberstyle={\scriptsize},
  stepnumber=1,
  numbersep=1zw,
  lineskip=-0.5ex
}

\begin{document}

\title{生体情報処理レポート課題\\
  MacCulloch-Pittsモデルについて}
\author{坪井正太郎(101830245)}
\date{\today}
\maketitle
\section{}
MacCulloch-Pittsモデルは、脳のニューロンの動きを単純にモデル化したものである。
このモデルの入出力値は、0か1で、N入力1出力である。
それぞれの入力に重みをつけた和が、しきい値を超えていれば1を、超えていなければ0を出力する。

これは、脳のニューロンが他の神経細胞から信号をうけとり、しきい値で発火する現象を模している。

出力値は、
\[y=f
  \left(
  \sum_{i=1}^{n}(w_ix_i)-\theta
  \right)
\]
ただし、
\[f(u)=
  \begin{cases}
    1 & (u>0)      \\
    0 & (u\leq  0)
  \end{cases}
\]
となる。

このモデルの入出力をつなげることで、単純パーセプトロンや、ニューラルネットワークを形成することができる。

モデル単体では、限定的な論理演算を記述することができる能力しかない。
これを1層重ねると線形分離可能な問題が、複数層では非線形な分類を行うことができる。

\section{論理演算の実現}
\subsection{not}
not演算は、
\[w=-1,\theta =-0.5\]
で実現できる。

\subsection{or}
or演算は
\[w[2]=\{1,1\},\theta =0.5\]
で実現できる。

\subsection{and}
or演算は
\[w[2]=\{1,1\},\theta =1.5\]
で実現できる。

これらの演算が実現できる一方で、xorなどの単純な直線で分類できない演算を行う能力はない。

\end{document}
