\documentclass[a4paper,15pt]{jsarticle}

% 数式
\usepackage{amsmath,amsfonts}
\usepackage{bm}
% 画像
\usepackage[dvipdfmx]{graphicx}

\usepackage{listingsutf8,jlisting} %日本語のコメントアウトをする場合jlistingが必要
%ここからソースコードの表示に関する設定
\lstset{
  basicstyle={\ttfamily},
  identifierstyle={\small},
  commentstyle={\smallitshape},
  keywordstyle={\small\bfseries},
  ndkeywordstyle={\small},
  stringstyle={\small\ttfamily},
  frame={tb},
  breaklines=true,
  columns=[l]{fullflexible},
  numbers=left,
  xrightmargin=0zw,
  xleftmargin=3zw,
  numberstyle={\scriptsize},
  stepnumber=1,
  numbersep=1zw,
  lineskip=-0.5ex
}

\begin{document}
\title{データベース演習課題2}
\author{坪井正太郎(101830245)}
\date{\today}
\maketitle
\subsection*{問1}
\Large\[\pi _{商品ID, 商品名}(\sigma _{価格<'10000'(商品)})\]

\subsection*{問2}
\Large\[\pi _{商品ID, 日付}(\sigma _{氏名='田中一郎'} (顧客 \Join 購入))\]

\subsection*{問3}
\Large\[\pi _{商品ID, 商品名}(\sigma _{価格>'50000'} (商品 \Join 顧客 \Join 購入))\]

\subsection*{問4}
\Large\[\pi _{商品ID, 商品名}(商品 \Join _{価格>価格10001} \delta _{価格\leftarrow 価格10001}(\pi _{商品ID}(\sigma _{='10001'}(商品))))\]

\subsection*{問5}
\Large\[\pi _{分類}(商品) - \pi _{分類} (商品 - \sigma _{価格>='10000'}(商品))\]

\end{document}
