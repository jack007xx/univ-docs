\documentclass[a4paper,10pt]{jsarticle}

% 数式
\usepackage{amsmath,amsfonts}
\usepackage{bm}
% 画像
\usepackage[dvipdfmx]{graphicx}
\usepackage{here}

\usepackage{listingsutf8,jlisting} %日本語のコメントアウトをする場合jlistingが必要
%ここからソースコードの表示に関する設定
\lstset{
  basicstyle={\ttfamily},
  identifierstyle={\small},
  commentstyle={\smallitshape},
  keywordstyle={\small\bfseries},
  ndkeywordstyle={\small},
  stringstyle={\small\ttfamily},
  frame={tb},
  breaklines=true,
  columns=[l]{fullflexible},
  numbers=left,
  xrightmargin=0zw,
  xleftmargin=3zw,
  numberstyle={\scriptsize},
  stepnumber=1,
  numbersep=1zw,
  lineskip=-0.5ex
}

\begin{document}

\title{データベース演習課題レポート}
\author{坪井正太郎(101830245)}
\date{\today}
\maketitle
\begin{enumerate}
  \item ページA,Bとaを読み込む\\

  \item aに含まれるレコードと、A,Bに含まれるレコードでマッチするものがあれば出力
  \item aの部分にbを読み込む
  \item bに含まれるレコードと、A,Bに含まれるレコードでマッチするものがあれば出力
  \item bの部分にcを読み込む\\
  \[\vdots \]
  \item fの部分にgを読み込む
  \item gに含まれるレコードと、A,Bに含まれるレコードでマッチするものがあれば出力\\

  \item gを入れ替えずに、A,Bの部分にC,Dを読み込む
  \item gに含まれるレコードと、C,Dに含まれるレコードでマッチするものがあれば出力
  \item gの部分にfを読み込む
  \item fに含まれるレコードと、C,Dに含まれるレコードでマッチするものがあれば出力
  \item fの部分にeを読み込む\\
  \[\vdots \]
  \item bの部分にaを読み込む
  \item aに含まれるレコードと、A,Bに含まれるレコードでマッチするものがあれば出力
\end{enumerate}

合計で17ページ分読み込んだ
\end{document}
