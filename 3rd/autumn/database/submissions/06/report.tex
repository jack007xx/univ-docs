\documentclass[a4paper,10pt]{jsarticle}

% 数式
\usepackage{amsmath,amsfonts}
\usepackage{bm}
% 画像
\usepackage[dvipdfmx]{graphicx}

\usepackage{listingsutf8,jlisting} %日本語のコメントアウトをする場合jlistingが必要
%ここからソースコードの表示に関する設定
\lstset{
  basicstyle={\ttfamily},
  identifierstyle={\small},
  commentstyle={\smallitshape},
  keywordstyle={\small\bfseries},
  ndkeywordstyle={\small},
  stringstyle={\small\ttfamily},
  frame={tb},
  breaklines=true,
  columns=[l]{fullflexible},
  numbers=left,
  xrightmargin=0zw,
  xleftmargin=3zw,
  numberstyle={\scriptsize},
  stepnumber=1,
  numbersep=1zw,
  lineskip=-0.5ex
}

\begin{document}

\title{演習課題6}
\author{坪井正太郎(101830245)}
\date{\today}
\maketitle
\section{}
候補キーは
\[\{B,C\},\{B,F\}\]

\section{}
\[R1\cap R2 = \{B,C\}\]
\[R1 - R2 = \{A,B\}\]
\[R2 - R1 = \{D,F\}\]
となる。

$C\rightarrow D\rightarrow A$より、$R1\cap R2 \rightarrow R1 - R2$が成立。
よって、この分解は無損失分解である。

\section{}
分解されたリレーションと、射影した従属性からは、$D\rightarrow A$の従属性を導くことができない。

よって、この分解は従属性保存分解ではない。

\section{}
関数従属性は既に極小被覆なので、各従属について、含まれる属性を分解にする。
\[\rho = \{\{A,B,E\},\{B,C,F\},\{C,D\},\{D,A\},\{F,C\}\}\]
部分集合として含まれる部分を除外して、
\[\{\{A,B,E\},\{B,C,F\},\{C,D\},\{D,A\}\}\]

\section{}
3NFの条件2に対応する部分を更に分解した。
\[\{\{A,B,E\},\{B,C\},\{C,F\},\{C,D\},\{D,A\}\}\]

\end{document}
