\documentclass[a4paper,10pt]{jsarticle}

% 数式
\usepackage{amsmath,amsfonts}
\usepackage{bm}
% 画像
\usepackage[dvipdfmx]{graphicx}
\usepackage{url}

\usepackage{listingsutf8,jlisting} %日本語のコメントアウトをする場合jlistingが必要
%ここからソースコードの表示に関する設定
\lstset{
  basicstyle={\ttfamily},
  identifierstyle={\small},
  commentstyle={\smallitshape},
  keywordstyle={\small\bfseries},
  ndkeywordstyle={\small},
  stringstyle={\small\ttfamily},
  frame={tb},
  breaklines=true,
  columns=[l]{fullflexible},
  numbers=left,
  xrightmargin=0zw,
  xleftmargin=3zw,
  numberstyle={\scriptsize},
  stepnumber=1,
  numbersep=1zw,
  lineskip=-0.5ex
}

\begin{document}

\title{先端計算機アーキテクチャレポート\\
  Armアーキテクチャの構造}
\author{坪井正太郎(101830245)}
\date{\today}
\maketitle
\section{概要}
このレポートでは、社会で活用されている計算機アーキテクチャとして、Armアーキテクチャについて記述する。
まず、社会における活用の例について書き、その後MIPSとの違いを中心にアーキテクチャの構造を分析する。

Armアーキテクチャの中でも、コアの実装、拡張に複数のラインナップがある。
ここでは、実装を限定せずに、どのような拡張がアーキテクチャに含まれるのか記述する。
基本的に、Armv8アーキテクチャを前提とした。

\section{社会における活用}
\subsection{ライセンス}
Armアーキテクチャは、他のアーキテクチャとは異なり、ARMによってライセンスを受けることで利用できる。
ライセンスを受けたプロセッサベンダは、提供を受けた設計に従ってプロセッサを製造することができる。

ライセンスには、命令セットレベルで互換性があるプロセッサを製造できるものや、RTL記述が提供されるもの、ネットリストからの配線をしてあるものまである。
命令レベルのライセンスは、用途に最適化できる部分が大きいが、設計の負担や製造までの時間がかかる。
一方でハードに近い設計の提供を受けると、そのような問題が少ない。
このように、ライセンスの選択によってベンダの都合に合った設計手法を選択することができる。

\subsection{設計方針}
Armアーキテクチャは、ハードウェアを小さくすることで消費電力を削減している。
そのため、携帯機器や組み込み向けのアーキテクチャとして採用されることが多い。
具体的には、キャッシュの容量や、パイプラインの段数を削減することでハードウェア要素を削り、省電化している。

一方で、高度なアプリケーションを実行するための拡張を一部では導入し、コード密度や処理性能を向上させている。
例として、SIMD命令、64bit命令のサポートの他に、Javaのバイトコードを直接実行する機能をサポートしている。

この2つを両立させるための設計として、少電力のコアと、高性能のコアを組み合わせる実装が設計され、big.LITTLEという名前で実現されている。

\section{アーキテクチャの特徴、構造}
\subsection{cpsrによる条件フラグ}
cpsr(プログラムステータスレジスタ)という条件フラグが格納されたレジスタを参照することで、1命令単位で命令の実行、非実行を制御することができる。
たとえば、EQ,NE命令や演算によるオーバーフロー、演算結果の符号によってcpsrの値を更新できる。
ARM命令は、全ての命令に条件フィールドがあるので、全ての命令実行をプログラムステートによって制御できる。

さらにcpsrは、プロセッサの実行状態や、エンディアン形式などのフラグを持っている。

少数命令のみの分岐や、後続の処理は必ず実行するif-then構造のように、コード密度や、分岐予測の失敗時にパイプラインに与える影響がある記述を削減することができる。

\subsection{Thumb/Thumb-2命令}
32bitのARM命令セットから、さらにコード密度を向上させ、省電力化するために、16bitのThumb命令が定義され、ARM命令セットと相互に呼び出し可能としている。
これは、頻繁に実行される手続きをARM命令で記述し、あまり頻繁に実行されない手続きでは全体のコード密度を向上させるためにThumb命令で記述する、といった利用がされる。
実際に、コードベースで少ない部分が、実行時間のほとんどを占めるので、この削減には効果がある。

また、Thumbの拡張で、Thumb-2が設計された。
これは、ARM←→Thumbが別のモードで実行することによるオーバーヘッドを、32bitのThumb-2命令を実行することで、削減される。

\subsection{Jazelle}
Armは、3つ目の命令セットとして、Javaのバイトコードをプロセッサ上の専用デコーダで解して、直接ほとんどの命令を実行することができる。
csprのフラグにモードの選択ビットがあるので、切替命令によってARM命令→Javaバイトコードに命令セットを変更することができる。

\subsection{DSP}
一部のArmプロセッサには、SIMD演算を行うことができる演算器が搭載されている。
大量のデータに対して、一度に同じ演算を高速に行うことができる。

命令セットとして、NEON命令セットが用意されている。
NEON用のレジスタ幅に対して、処理するデータを決めることで、一度に対象になるデータ数が決まる。
例えば、64bitのレジスタ2本を16bitデータの加算オペランドにとると、1度に4つのデータに対して加算処理をして、64bitディスティネーションレジスタにできる。

NEONによるベクタ処理は、マルチメディア処理の他に、マシンラーニングの目的でも広範に利用される。
特に、高い精度を要求されない処理に対応するために、高速、省データの半精度浮動小数点演算がサポートされた。

今回調査した中では、ArmのDSPを利用してマシンラーニングを行っている例を見つけることができなかった。
しかし、QualcomのSnapDragonシリーズには、Qualcomが内製したHexagonというDSPが搭載されているなど、各ベンダによってDSPが追加されている。

Hexagonが、NEON命令セットをサポートしているかについては、今回の調査で資料を見つけることができなかった。
DSPを使った開発にはHexagon DSP SDKが、SD上のマシンラーニング向け開発は、SNPE(Snapdragon Neural Processing Engine)を使用して開発される。
SNPEでは、コンパイルはArmをターゲットとしていて、ソース内でアクセラレータを指定するようになっていたので、おそらくNEON命令を実行するDSPとして、Hexagonが搭載されていると予想できる。

\begin{thebibliography}{9}
  \bibitem{キー2} David A.Patterson/John L.Hennessy コンピュータの構成と設計(上), 付録E
  \bibitem{キーN} Hexagon DSP SDK, \url{https://developer.qualcomm.com/software/hexagon-dsp-sdk}
\end{thebibliography}

\end{document}
