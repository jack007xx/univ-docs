\documentclass[a4paper,10pt]{jsarticle}

% 数式
\usepackage{amsmath,amsfonts}
\usepackage{bm}
% 画像
\usepackage[dvipdfmx]{graphicx}
\usepackage{here}

\usepackage{listingsutf8,jlisting} %日本語のコメントアウトをする場合jlistingが必要
%ここからソースコードの表示に関する設定
\lstset{
  basicstyle={\ttfamily},
  identifierstyle={\small},
  commentstyle={\smallitshape},
  keywordstyle={\small\bfseries},
  ndkeywordstyle={\small},
  stringstyle={\small\ttfamily},
  frame={tb},
  breaklines=true,
  columns=[l]{fullflexible},
  numbers=left,
  xrightmargin=0zw,
  xleftmargin=3zw,
  numberstyle={\scriptsize},
  stepnumber=1,
  numbersep=1zw,
  lineskip=-0.5ex
}

\begin{document}

\title{計算理論追加課題}
\author{坪井正太郎(101830245)}
\date{\today}
\maketitle
\section{2-PDAでのDTMの模倣}
第1スタックでヘッダの位置より左側の入力列、第2スタックでヘッダの指す記号とそれより右側の入力列を表す。
動作開始時に、模倣するDTMの初期条件に遷移し、第2スタックに入力列を全て積む。
以後、2-PDAの$\epsilon$遷移で、左右の遷移を模倣していく。

右に遷移するとき、第2スタックのトップを見ても次の記号がわからないので、1つ取り出して第1スタックにそのまま積んで、第2スタックを参照する。
第2スタックから1つpopして、書き換えがあれば書き換えて、なければそのままpushする。

左に遷移するときは第1スタックに対して、1つpopして、書き換えがあれば書き換えて、なければそのままpushする動作を行う。

状態は、DTMのものをそのまま模倣して、受理状態になればPDAでも受理する。

\section{DTMでk-PDAを模倣する($k\ge 0$)}
通常のPDAで使うアルファベット、DTMの終端記号に加えて、入力部とスタック部の区切りのために記号を追加する。
また、模倣するPDAで指す記号を記憶するために、そのための記号を追加する
以下のようなDTMを作り、k-PDAを模倣する。

\[...\sqcup \sqcup \{位置記憶\}\{PDAへの入力列\}\{区切り記号\}\{第1スタック\}\{区切り記号\}...\{区切り記号\}\{第kスタック\}\sqcup \sqcup ...\]

各スタック部の一番左を対応するスタックのトップと考える。

ここで、DTMは記号の挿入操作、削除操作を行うことができるので、スタックのpush,popが模倣できる。
また、適宜位置記号を右に動かすことで入力記号を受け付けることもできる。

状態は、PDAのものをそのまま模倣し、PDAでの受理状態に遷移したらDTMでも受理する。\

以上の操作で、k-PDAをそのまま模倣することができる。

\section{k-PDAと2-PDAの等価性($k\ge 3$)}
2-PDAで「k-PDAを模倣するDTM」を模倣できる。(2-PDAでk-PDAを模倣することができる)
また、k-PDAで2-PDAを模倣することは明らかに可能である。

このことから、k-PDAと2-PDAは等価であると言える。

\end{document}
