\documentclass[a4paper,10pt]{jsarticle}

% 数式
\usepackage{amsmath,amsfonts}
\usepackage{bm}
% 画像
\usepackage[dvipdfmx]{graphicx}
\usepackage{here}

\usepackage{listingsutf8,jlisting} %日本語のコメントアウトをする場合jlistingが必要
%ここからソースコードの表示に関する設定
\lstset{
  basicstyle={\ttfamily},
  identifierstyle={\small},
  commentstyle={\smallitshape},
  keywordstyle={\small\bfseries},
  ndkeywordstyle={\small},
  stringstyle={\small\ttfamily},
  frame={tb},
  breaklines=true,
  columns=[l]{fullflexible},
  numbers=left,
  xrightmargin=0zw,
  xleftmargin=3zw,
  numberstyle={\scriptsize},
  stepnumber=1,
  numbersep=1zw,
  lineskip=-0.5ex
}

\begin{document}

\title{計算理論第2回課題レポート}
\author{坪井正太郎(101830245)}
\date{\today}
\maketitle
第2テープで、1つの追記($n^2$個目を書き加えた時点)を終えるまで、第1テープは開始記号で止まっておく。
第2テープで1つの追記が終わったら、第2テープも開始記号まで戻し、第1テープと第2テープを一緒に右にシフトしていく。

第1テープで先に$\sqcup$にあたった場合は非受理、第2テープで先にあたった場合は2つとも開始記号まで戻し、第2テープで次の追記を行う。
同時に$\sqcup$にあたった場合には受理する。
\\

\subsubsection*{多テープDTMの動作関数に関して補足}
講義資料では、多テープDTMの動作関数の定義は何もしない動作(LにもRにも移動しない)を許していない。
ここで、1つのテープ上でLR交互に移動することと、有限個の入力を状態に繰り込むことを考えると、何もしない動作を許す多テープDTMは、スライドで定義されている多テープDTMでエミュレートできる。
したがって、第1テープにおいて開始記号で止まっておくような動作は可能。

\end{document}
