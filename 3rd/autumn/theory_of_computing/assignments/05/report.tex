\documentclass[a4paper,10pt]{jsarticle}

% 数式
\usepackage{amsmath,amsfonts}
\usepackage{bm}
% 画像
\usepackage[dvipdfmx]{graphicx}
\usepackage{here}

\usepackage{listingsutf8,jlisting} %日本語のコメントアウトをする場合jlistingが必要
%ここからソースコードの表示に関する設定
\lstset{
  basicstyle={\ttfamily},
  identifierstyle={\small},
  commentstyle={\smallitshape},
  keywordstyle={\small\bfseries},
  ndkeywordstyle={\small},
  stringstyle={\small\ttfamily},
  frame={tb},
  breaklines=true,
  columns=[l]{fullflexible},
  numbers=left,
  xrightmargin=0zw,
  xleftmargin=3zw,
  numberstyle={\scriptsize},
  stepnumber=1,
  numbersep=1zw,
  lineskip=-0.5ex
}

\begin{document}

\title{計算理論第5回課題}
\author{坪井正太郎(101830245)}
\date{\today}
\maketitle
\section{}
解は$3,2,2,1$で一致列は$1010101011$となる。

\section{}
(a),(c)は正しくないが、(b)は正しい。
PCPのインスタンス(C,D)の解が1から始まる場合、その解はMPCPでの解にもなる。
よって(c)は正しくない。

逆に、1から始まらない場合、MPCPでの解にはなり得ない。
よって(a)も正しくない。

以上より、(b)のみが正しいと言える。

\section{}
思いつきませんでした。

\end{document}
