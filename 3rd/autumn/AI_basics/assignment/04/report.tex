\documentclass[a4paper,10pt]{jsarticle}

% 数式
\usepackage{amsmath,amsfonts}
\usepackage{bm}
% 画像
\usepackage[dvipdfmx]{graphicx}
\usepackage{here}

\usepackage{url}

\usepackage{listingsutf8,jlisting} %日本語のコメントアウトをする場合jlistingが必要
%ここからソースコードの表示に関する設定
\lstset{
  basicstyle={\ttfamily},
  identifierstyle={\small},
  commentstyle={\smallitshape},
  keywordstyle={\small\bfseries},
  ndkeywordstyle={\small},
  stringstyle={\small\ttfamily},
  frame={tb},
  breaklines=true,
  columns=[l]{fullflexible},
  numbers=left,
  xrightmargin=0zw,
  xleftmargin=3zw,
  numberstyle={\scriptsize},
  stepnumber=1,
  numbersep=1zw,
  lineskip=-0.5ex
}

\begin{document}

\title{人工知能基礎レポート課題\\
安全な車を実現するためになにをするか}
\author{坪井正太郎(101830245)}
\date{\today}
\maketitle
\section{安全のために必要な技術}
安全を確保するための技術を大きく、「車の状態を認識する技術」、「周囲の状態を認識する技術」、「状況からどのような操作を行うか決定し、実際に制御する技術」、と分けた。
このうちのどれが欠けても安全な車を作ることはできず、逆に安全な車を作るための技術は、このうちのどれかに分類されると考えたからである。
以下ではそれぞれの分類について、必要だと考えた具体的な技術を説明する。

\subsection{車の状態を認識する技術}
車載のコンピュータが車の状態を認識することは、車の制御に必要不可欠である。
現在の車でも、ABSやクルーズコントロールのために車の各部の状態をセンシングしていおり、車自体の制御に必要な情報を収集する分には十分であると考えた。

しかし、将来の車の安全性を確保するためには、乗車している人間の状態を把握することも必要であると考えられる。
具体的には、注視すべき方向を向いているか、意識が失われていないか、という情報を知っておかなければ、十分に安全性を確保できない。

\subsection{周囲の状態を認識する技術}
授業内で扱われていたLiDARのように、周囲の物体を認識する技術は自車の位置を知る目的、衝突の回避の目的でも必要である。

授業では自律探査ロボットを利用した死角のセンシング技術が紹介されていたが、それらの発展として、周囲の車との3次元地図の共有を考えた。
周辺の車と3次元地図の情報を補い合うことで、死角の解消と注意区間の検出を自律探査ロボットを使わずに実現できる。
これによって、授業内で言われていた自律探査ロボットと人の衝突も防ぐことができると考えた。

\subsection{状況からどのような操作を行うか決定し、実際に制御する技術}
安全な運転を行うために、車は3次元地図上での安全な経路を選択し、そのルート通りの移動を制御する。
また、突発的な危険(運転者の問題、注意区間での衝突)を回避してルートを書き直す、速度を落とすなどの制御が必要である。

これらを実現させるためには、自己位置の推定、注意区間の検出のような要素技術が必要であると考えた。

\section{それらの技術の実装}
今回考察した技術の中で実装されていないと考えたのは、「乗車している人間の状態認識」、「周囲の車との地図共有」であり、それらの実装について考察した。

\subsection{乗車している人間の状態認識}
調査したところ、搭乗者の意識状態を認識して異常があった場合に周囲の車に伝える技術は、既に市販の車にも実装されていることが判明した。
調査した中では、国交省がガイドラインを発表している\cite{kokko}他、昨年10月に発売されたスバルのレヴォーグに搭載されたアイサイトXに、該当する機能が実装されていた\cite{aisaito}。

また、例として挙げた注視方向の検知についても、アイトラッキング技術として実用化されている。

\subsection{周囲の車との地図共有}
複数の移動している車による地図共有では、互いの地図における相対位置合わせが問題となると考えた。
相対位置の特定のためには、GPSなどで絶対座標を取得する方法がまず考えられるが、お互いから見える範囲(センサーの検出範囲)に共有範囲を絞ることで、相対位置を算出することもできるのではないかと考えた。

絶対座標による地図合成のほうが、より厳密な地図を作成することができる点で優位である。
しかし、例えば注意区間のおおよその位置を把握する目的に絞れば、相対位置が厳密でなくても安全性を高めることができると考えた。
実際、「○km先、事故車あり」などの標識による情報伝達を、もっと細かい単位で行えると考えると、効果はあると言える。

\section{安全性の評価}
通常の評価方法、例えばクローズドコースでのテストや、シミュレーションでの反復テストは当然行われるべきであるが、実際に車に搭乗するユーザーが安全性について評価できる仕組みが必要である。
前々回のレポートで考察した「自動運転システムの判断を評価するためのシステム」を用いれば、このような仕組みを作ることができると考えた。

このシステムでは、VR空間で人工知能の判断について安全に検証を行うことができる。
これによって、メーカーだけでなくユーザーも車の安全性に対する評価を行うことができ、安心して搭乗することができる。

\begin{thebibliography}{9}
  \bibitem{kokko}ドライバー異常時対応システムについて\url{https://www.mlit.go.jp/common/001301889.pdf}
  \bibitem{aisaito}アイサイト X \url{https://www.subaru.jp/levorg/levorg/safety/safety2_2}
\end{thebibliography}

\end{document}
